\documentclass{book}
\usepackage{fontspec}
\usepackage{xunicode}
\usepackage[english]{babel}
\usepackage{fancyhdr}
\usepackage[htt]{hyphenat}
\usepackage[a5paper, top=2cm, bottom=1.5cm, left=2.5cm,right=1.5cm]{geometry}
\makeatletter
\date{}
\pagestyle{fancy}
\fancyhead{}
\fancyhead[CO,CE]{\thepage}
\fancyfoot{}
\makeatother
\title{Economic Democracy}
\author{C. H. Douglas}
\begin{document}
\thispagestyle{empty}
\vspace*{\stretch{1}}
\begin{center}
	{\Huge \@title   \\[5mm]}
\end{center}
\vspace*{\stretch{2}}
\newpage
\thispagestyle{empty}
\cleardoublepage
\begin{center}
	\thispagestyle{empty}
	\vspace*{\baselineskip}
	\rule{\textwidth}{1.6pt}\vspace*{-\baselineskip}\vspace*{2pt}
	\rule{\textwidth}{0.4pt}\\[\baselineskip]
	{\Huge\scshape \@title   \\[5mm]}
	{\Large }
	\rule{\textwidth}{0.4pt}\vspace*{-\baselineskip}\vspace{3.2pt}
	\rule{\textwidth}{1.6pt}\\[\baselineskip]
	\vspace*{4\baselineskip}
	{\Large \@author}
	\vfill
\end{center}
\pagebreak
\newpage
\thispagestyle{empty}
\null\vfill
\noindent
\begin{center}
	{\emph{\@title}, © \@author.\\[5mm]}
	{This work is free of known copyright restrictions.\\[5mm]}
\end{center}
\pagebreak
\newpage
\setcounter{tocdepth}{0}
\setcounter{secnumdepth}{0}

\chapter*{Preface}
\label{chapter-0}
Written for the most part under the pressure of War conditions, this book is an attempt to disentangle from a mass of superficial features such as Profiteering, and alleged scarcity of commodities, a sufficient portion of the skeleton of the Structure we call Society as will serve to suggest sound reasons for the decay with which it is now attacked; and afterwards to indicate the probable direction of sound and vital reconstruction.

My apologies and sympathy are offered to the reader in respect of the severe concentration which its tabloid treatment of technical methods demands; but I have some grounds for supposing that the matter it contains has aroused sufficient interest to excuse its presentation in this form.

I am indebted to my friend Mr. A. R. Orage, the Editor of \emph{The New Age} (in which review, together with the remainder of the book, it first appeared) for the use of the block which forms the frontispiece.

C. H. Douglas.

Heath End, Basingstoke. \emph{November}, 1919.

\chapter{Chapter One}
\label{chapter-1}
There has been a very strong tendency, fortunately not now so strong as it was, to regard fidelity to one set of opinions as being something of which to be proud, and consistency in the superficial sense as a test of character.

The Scottish political constituent who always voted for a Liberal because he was too Conservative to change, has his counterpart in every sphere of human activity, and most particularly so in that of economics, where the tracing back to first principles of the dogmas used for everyday purposes requires, in addition to some little aptitude and research, a laborious effort of thought and logic very foreign to our normal methods.

It thus comes about that modification in the creed of the orthodox is both difficult and conducive to exasperation; since because the form is commonly mistaken for the substance it is not clearly seen why a statement which has embodied a sound principle, may in course of time become a dangerous hindrance to progress.

Of such a character are many of our habits of thought and speech to-day. Because from the commercial policy of the nineteenth century has quite clearly sprung great advance in the domain of science and the mastery of material nature, the commercialist, quite honestly in many cases, would have us turn the land into a counting house and drain the sea to make a factory. On the other hand the Social Reformer, obsessed as well he might be, with the poverty and degradation which shoulder the very doors of the rich, is apt to turn his eyes back to the days antecedent to the Industrial Revolution note, or assume, that the conditions he deplores did not exist then, at any rate, in so desperate a degree; and condemn all business as abominable.

At various well-defined epochs in the history of civilisation there has occurred such a clash of apparently irreconcilable ideas as has at this time most definitely come upon us. Now, as then, from every quarter come the unmistakable signs of crumbling institutions and discredited formulae, while the wide-spread nature of the general unrest, together with the immense range of pretext alleged for it, is a clear indication that a general re-arrangement is imminent.

As a result of the conditions produced by the European War, the play of forces, usually only visible to expert observers, has become apparent to many who previously regarded none of these things. The very efforts made to conceal the existence of springs of action other than those publicly admitted, has riveted the attention of an awakened proletariat as no amount of positive propaganda would have done. A more or less conscious effort to refer the results of the working of the social and political system to the Bar of individual requirement has, on the whole, quite definitely resulted in a verdict for the prosecution; and there is little doubt that sentence will be pronounced and enforced.

Before proceeding to the consideration of the remedies proposed, it may be well to emphasise the more salient features of the indictment, and in doing this it is of the first consequence to make very sure of the code against which the alleged offences have been committed. And here we are driven right back to first principles—to an attempt to define the purposes, conscious or unconscious, which govern humanity in its ceaseless struggle with environment.

To cover the whole of the ground is, of course, impossible. The infinite combinations into which the drive of evolution can assemble the will, emotions and desires, are probably outside the scope of any form of words not too symbolical for everyday use.

But of the many attempts which have been made it is quite possible that the definition embodied in the majestic words of the American Declaration of Independence, “the inalienable right of man to life, liberty and the pursuit of happiness “is still unexcelled, although the promise of its birth is yet far from complete justification; and if words mean anything at all, these words are an assertion of the supremacy of the individual considered collectively, over any external interest. Now, what does this mean? First of all, it does \emph{not} mean anarchy, nor does it mean exactly what is commonly called individualism, which generally resolves itself into a claim to force the individuality of others to subordinate itself to the will-to-power of the self-styled individualist. And most emphatically it does not mean collectivism in any of the forms made familiar to us by the Fabians and others.

It is suggested that the primary requisite is to obtain in the re-adjustment of the economic and political structure such control of initiative that by its exercise every individual can avail himself of the benefits of science and mechanism; that by their aid he is placed in such a position of advantage, that in common with his fellows he can choose, with increasing freedom and complete independence, whether he will or will not assist in any project which may be placed before him.

The basis of independence of this character is most definitely economic; it is simply hypocrisy, conscious “or unconscious, to discuss freedom of any description which does not secure to the individual, that in return for effort exercised as a right, not as a concession, an average economic equivalent of the effort made shall be forthcoming.

It seems clear that only by a recognition of this necessity can the foundations of society be so laid that no superstructure built upon them can fail, as the superstructure of capitalistic society is most unquestionably failing, because the pediments which should sustain it are honeycombed with decay.

Systems were made for men, and not men for systems, and the interest of man which is self-development, is above all systems, whether theological, political or economic.

\chapter{Chapter Two}
\label{chapter-2}
Accepting this statement as a basis of constructive effort, it seems clear that all forms, whether of government, industry or society must exist contingently to the furtherance of the principles contained in it. If a State system can be shown to be inimical to them—it must go; if social customs hamper their continuous expansion—they must be modified; if unbridled industrialism checks their growth, then industrialism must be reined in. That is to say, we must build up from the individual, not down from the State.

It is necessary to be very clear in thus defining the scope of our inquiry since the exaltation of the State into an authority from which there is no appeal, the exploitation of a public opinion which at the present time is frequently manufactured for interested purposes, and other attempts to shift the centre of gravity of the main issues; these are all features of one of the policies which it is our purpose to analyse. If, therefore, any condition can be shown to be oppressive to the individual, no appeal to its desirability in the interests of external organisation can be considered in extenuation; and while cooperation is the note of the coming age, our premises require that it must be the cooperation of reasoned assent, not regimentation in the interests of any system, however superficially attractive.

There is no doubt whatever that a mangled and misapplied Darwinism has been one of the most potent factors in the social development of the past sixty years; from the date of the publication of “The Origin of Species” the theory of the “survival of the fittest” has always been put forward as an omnibus answer to any individual hardship; and although such books as Mr. Benjamin Kidd’s “Science of Power” have pretty well exposed the reasons why the individual, efficient in his own interest and consequently well-fitted to survive, may and will possess characteristics which completely unfit him for positions of power in the community, we may begin our inquiry by noticing that one of the most serious causes of the prevalent dissatisfaction and disquietude is the obvious survival, success and rise to positions of great power, of individuals to whom the term '\emph{fittest}' could only be applied in the very narrowest sense. And in admitting the justice of the criticism, it is not of course necessary to question the soundness of Darwin’s theory. Such an admission is simply evidence that the particular environment in which the “fittest” are admittedly surviving and succeeding is unsatisfactory; that in consequence those best fitted for it are not representative of the ideal existent in the mind of the critic, and that environment cannot be left to the unaided law of Darwinian evolution, in view of its effect on other than material issues.

To what extent the rapid development of systematic organisation is connected with the statement of the law of biological evolution would be an interesting speculation; but the second great factor in the changes which have been taking place during the final years of the epoch just closing is undoubtedly the marshalling of effort in conformity with well-defined principles, the enunciation of which has largely proceeded from Germany, although their source may very possibly be extra-national; and while these principles have been accepted and developed in varying degree by the governing classes of all countries, the dubious honour of applying them with rigid logic and a stern disregard of by-products, belongs without question, to the land of their birth. They may be summarised as a claim for the complete subjection of the individual to an objective which is externally imposed on him; which it is not necessary or even desirable that he should understand in full; and the forging of a social, industrial and political organisation which will concentrate control of policy while making effective revolt completely impossible, and leaving its originators in possession of supreme power.

This demand to subordinate individuality to the need of some external organisation, the exaltation of the State into an authority from which there is no appeal (as if the State had a concrete existence apart from which those who operate its functions), the exploitation of “public opinion” manipulated by a Press owned and controlled from the apex of power, are all features of a centralising policy commended to the individual by a claim that the interest of the community is thereby advanced, and its results in Germany have been nothing less than appalling. The external characteristics of a nation with a population of 65 millions have been completely altered in two generations, so that from the home of idealism typified by Schiller, Goethe, and Heine, it has become notorious for bestiality and inhumanity only offset by a slavish discipline. Its statistics of child suicide during the years preceding the war exceeded by many hundreds percent, those of any other country in the world, and were rising rapidly. Insanity and nervous breakdown were becoming by far the gravest problem of the German medical profession. Its commercial morality was devoid of all honour, and the external influence of Prussian ideals on the world has undoubtedly been to intensify the struggle for existence along lines which quite inevitably culminated in the greatest war of all history.

The comparative rapidity with which the processes matured was no doubt aided by an essential servility characteristic of the Teutonic race, and the attempt to embody these principles in Anglo-Saxon communities has not proceeded either so fast or so far; but every indication points to the imminence of a determined effort to transfer and adopt the policy of central, or, more correctly, pyramid, control from the nation it has ruined to others, so far more fortunate.

Thus far we have examined the psychological aspect of control exercised through power. Let us turn for a moment to its material side. Inequalities of circumstance confront us at every turn. The vicious circles of unemployment, degradation and unemployability, the disparity between the reward of the successful stock-jobber and the same man turned private soldier, enduring unbelievable discomfort for eighteen-pence per day, the gardener turned pieceworker, earning three times the pay of the skilled mechanic, are instances at random of the erratic working of the so-called law of supply and demand.

In the sphere of politics it is clear that all settled principle other than the consolidation of power, has been abandoned, and mere expediency has taken its place. The attitude of statesman and officials to the people in whose interests they are supposed to hold office, is one of scarcely veiled antagonism, only tempered by the fear of unpleasant consequences. In the State services, the easy supremacy of patronage over merit, and vested interest over either, has kindled widespread resentment, levelled not less at the inevitable result, than at the personal injustice involved.

In its relations with labour, the State is hardly more happy. In the interim report of the Commission on Industrial Unrest, the following statement occurs:-—

\begin{quotation}\
	There is no doubt that one cause of labour unrest is that workmen have come to regard the promises and pledges of Parliament and Government Departments with suspicion and distrust.
\end{quotation}

In industry itself, the perennial struggle between the forces of Capital and Labour, on questions of wages and hours of work, is daily becoming complicated by the introduction of fresh issues such as welfare, status and discipline, and it is universally recognised that the periodic strikes which convulse one trade after another, have common roots far deeper than the immediate matter of contention. In the very ranks of Trade Unionism, whose organisation has become centralised in opposition to concentrated capital, cleavage is evident in the acrimonious squabbles between the skilled and the unskilled, the rank and file and the Trade Union official.

Although the diversion of the forces of industry to munition work of, in the economic sense, an unreproductive character has created an almost unlimited outlet for manufactures of nearly every kind, it is not forgotten that before the war the competition for markets was of the fiercest character and that the whole world was apparently overproducing; in spite of the patent contradiction offered by the existence of a large element of the population continually on the verge of starvation (Snowden Socialism and Syndicalism), and a great majority whose only interest in great groups of the luxury trades was that of the wage-earner.

The ever-rising cost of living has brought home to large numbers of the salaried classes problems which had previously affected only the wage-earner. It is realised that “labour-saving” machinery has only enabled the worker to do more work; and that the ever-increasing complexity of production, paralleled by the rising price of the necessaries of life, is a sieve through which out and for ever out go all ideas, scruples and principles which would hamper the individual in the scramble for an increasingly precarious existence.

We see, then, that there is cause for dissatisfaction with not only the material results of the economic and political systems, but that they result in an environment which is hostile to moral progress and intellectual expansion; and it will be noticed in this enumeration of social evils, which is only so wide as is necessary to suggest principles, that emphasis is laid on what may be called abstract defects and miscarriages of justice, as well as on the material misery and distress which accompany them. The reason for this is that the twin evil (common more or less to all existing organised Society) of servility is poverty, as has been clearly recognised by all shades of opinion amongst the exponents of Revolutionary Socialism. Poverty is in itself a transient phenomenon, but servility (not necessarily, of course, of manner) is a definite component of a system having centralised control of policy as its apex; and while the development of self-respect is universally recognised to be an antecedent condition to any real improvement in environment, it is not so generally understood that a world-wide system is thereby challenged. In referring the existent systems to the standard we have agreed to accept, however, it seems clear that the stimulation of independence of thought and action is a primary requirement, and to the extent to which these qualities are repressed, social and economic conditions stand condemned as undesirable.

Now it may be emphasised that a centralised or pyramid form of control may be, and is in certain conditions, the ideal organisation for the attainment of one specific and material end. The only effective force by which any objective can be attained is in the last analysis the human will, and if an organisation of this character can keep the will of all its component members focussed on the objective to be attained, the collective power available is centralised control of policy as its apex; and while the development of self-respect is universally recognised to be an antecedent condition to any real improvement in environment, it is not so generally understood that a world-wide system is thereby challenged. In referring the existent systems to the standard we have agreed to accept, however, it seems clear that the stimulation of independence of thought and action is a primary requirement, and to the extent to which these qualities are repressed, social and economic conditions stand condemned as undesirable.

To crystallise the matter into a phrase; in respect of any undertaking, centralisation is the way to do it, but is neither the correct method of deciding what to do or of selecting the individual who is to do it.

\chapter{Chapter Three}
\label{chapter-3}
We are thus led to inquire into environment with a view to the identification, if possible, of conditions to which can be charged the development of servility on the one hand, and the discouragement of possibly more desirable characteristics on the other, and in this inquiry it is necessary to avoid the real danger of mistaking effects for causes; and, further, to beware of seeing only one phenomenon when we are really confronted with several.

For instance, that from the misuse of the power of capital many of the more glaring defects of society proceed is certain, but in claiming that in itself the private \emph{administration} of industry is the whole source of these evils, the Socialist is almost certainly claiming too much, confounding the symptom with the disease, and taking no account of certain essential facts. It is most important to differentiate in this matter, between private enterprise utilising capital, and the abuse of it.

The private administration of capital has had a credit as well as a debit side to its account; without private enterprise backed by capital, scientific progress, and the possibilities of material betterment based on it, would never have achieved the rapid development of the past hundred years; and still more important at this time, only the control of capital, which on the one hand has degraded propaganda into one of the Black Arts, has, on the other, made possible such crusades against an ill-informed or misled public opinion as, for instance, the anti-slavery Campaign of the early nineteenth century, or the parallel activities of the anti-sweating league at the present day. The very agitation carried on against capitalism itself would be impossible without the freedom of action given by the private control of considerable funds.

The capitalistic system in the form in which we know it has served its purpose, and may be replaced with advantage; but in any social system proposed, the first necessity is to provide some bulwark against a despotism which might exceed that of the Trust, bad as the latter has become. In our anxiety to make a world safe for democracy it is a matter of real urgency that we do not tip out the baby with the bath water, and, by discarding too soon what is clearly an agency which can be made to operate both ways, make democracy even more unsafe for the individual than it is at present.

The danger which at the moment threatens individual liberty far more than any extension of individual enterprise is the Servile State; the erection of an irresistible and impersonal organisation through which the ambition of able men, animated consciously or unconsciously by the lust of domination, may operate to the enslavement of their fellows. Under such a system the ordinary citizen might, and probably would, be far worse off than under private enterprise freed from the domination of finance and regulated in the light of modern thought. The consideration of any return to isolated industrial undertakings is quite academic, since there is not the faintest probability of its occurrence, but that stage of development had undoubtedly certain valuable features which it would be well to preserve and revive. The large profit-making limited company which distributes its profits over a wide area is already rapidly displacing the family business and, as will be seen, it is not alone in the profit-making aspect of its activities that its worst features lie.

In attacking capitalism, collective Socialism has largely failed to recognise that the real enemy is the will-to-power, the positive complement to servility, of which Prussianism, with its theories of the supreme state and the unimportance of the individual (both of which are the absolute negation of private enterprise) is only the fine flower; and that nationalisation of all the means of livelihood, without the provision of much more effective safeguards than have so far been publicly evolved, leaves the individual without any appeal from its only possible employer and so substitutes a worse, because more powerful, tyranny for that which it would destroy.

It is a most astonishing fact that the experience of hundreds of thousands of men and women in such departments as the Post Office, where real discontent is probably more general, and the material and psychological justification for it more obvious, than in any of the more modern industrial establishments, has not been sufficient to impress the public with the futility of mere nationalisation. This is not in any sense a disparagement of the excellent qualities ot large numbers of Government officials; it is merely an attempt to indicate the remarkable facility with which well-intentioned people will allow themselves to be hypnotised by a phrase. It is notorious that the State Socialists of Germany, commonly known as the Majority Party, were of the greatest possible assistance to Junkerdom in carrying out its plans for a Prussian world hegemony; while in our own country the bureaucrat and the Fabian have, on the whole, not failed to understand each other; and the explanation is simply that both, either consciously or unconsciously, assume that there is no psychological problem involved in the control of industry just as the Syndicalist is, with more justification, apt to stress the psychological to the exclusion of the technical aspect.

Because the control of capital has given power, the effect of the operation of the will-to-power has been to accumulate capital in a few groups, possibly composed of large numbers of shareholders, but frequently directed by one man; and this process is quite clearly a stage in the transition from decentralised to centralised power. This centralisation of the power of capital and credit is going on before our eyes, both directly in the form of money trusts and bank amalgamations, and indirectly in the confederation of the producing industries representing the capital power of machinery. It has its counterpart in every sphere of activity: the coalescing of small businesses into larger, of shops into huge stores, of villages into towns, of nations into leagues, and in every case is commended to the reason by the plea of economic necessity and efficiency. But behind this lies always the will-to-power, which operates equally through politics, finance or industry, and always towards centralisation. If this point of view be admitted, it seems perfectly clear that to the individual it will make very little difference what name is given to centralisation. Nationalisation without decentralised control of policy will quite effectively instal the trust magnate of the next generation in the chair of the bureaucrat, with the added advantage to him that he will have no shareholders’ meeting.

One of the more obvious effects of the concentration of credit-capital in a few hands, which simply means the centralisation of directive power, is its contribution to the illusion of the fiercely competitive nature of international trade. Although as we shall see, in considering the economics of the increasing employment of machinery for productive purposes, this phenomenon has been confounded with one to which it is only indirectly connected, it may be convenient at this time to point out one method by which this illusion is produced, and it is probably not possible to do so in better words than those used by Mr. J. A. Hobson in his “Democracy After the War”:–

\begin{quotation}
	Where the product of industry and commerce is so divided that wages are low while profits, interest, and rent are relatively high, the small purchasing power of the masses sets a limit on the home market for most staple commodities. The staple manufacturers, therefore, working with modern mechanical methods, that continually increase the pace of output, are in every country compelled to look more and more to export trade, and to hustle and compete for markets in the backward countries of the world... Just as the home market was restricted by a distribution of wealth which left the mass of people with inadequate power to purchase and consume, while the minority who had the purchasing power either wanted to use it in other ways or to save it and apply it to an increased production which still further congested the home markets, so likewise with the world markets... Closely linked with this practical limitation of the expansion of markets for goods is the limitation of profitable fields of investment. The limitation of home markets implies a corresponding limitation in the investment of fresh capital in the trades supplying these markets.
\end{quotation}

Because capitalism per se is largely the instrument through which the will-to-power operates in the economic sphere, some examination of its methods is necessary. The accumulation of financial wealth through the making of profit is merely one of the uses or abuses of money, but it is in this sense that capitalism is associated to a very great extent in the popular mind with the processes of manufacture, production and distribution, and it is in this sense that the word is here employed. The capitalistic system is based fundamentally on the financial perversion of the law of supply and demand, which involves a claim that there exists an intrinsic relation between need or requirement, and legitimate price or exchange value; a statement which is becoming increasingly discredited, and is negatived in the limitation of monopoly values, by common consent, in respect of public utility companies, such as lighting, water and transportation undertakings.

Proceeding from an economic system based on this assumed relation, however, the capitalistic producer only parts with his product for a sum in excess of that representing its cost to him, receiving payment through the agency of money in its various forms of cash and financial credit, which, so far as they are convertible, have been defined as any medium which has reached such a degree of acceptability that no matter what it is made of, and no matter why people want it, no one will refuse it in exchange for his product. (Professor Walker, “Money, Trade and Industry,” p. 6).

So long as this definition holds good, it is obvious that the possession of money, or financial credit convertible into money, establishes an absolute lien on the services of others in direct proportion to the fraction of the whole stock controlled, and further that the whole stock of financial wealth, inclusive of credit, in the world should, by the definition, be sufficient to balance the aggregate book price of the world’s material assets and prospective production; and generally it is assumed that the banks regulate the figures of wealth by the creation of credits broadly representing the mobilisation value of these assets either in esse or in posse, such value being for financial purposes the transfer or selling price and bearing no relation to the usage value of the article so appraised.

But for reasons which will be evident in considering the costing of production at a later stage of our inquiry, the book value of the world’s stocks is always greater than the apparent financial ability to liquidate them, because these book values already include mobilised credits; the creation of subsidiary financial media, in the form of further bank credits, becomes necessary, and results in the piling up of a system on figures which the accountant calls capital, but which are in fact merely a function of prices. The effect of this is, of course, to decrease progressively the purchasing power of money, or, in other words, to concentrate the lien on the services of others, which money gives, in the hands of those whose rate of increase is most rapid. Intrinsic improvements in manufacturing methods operate to delay this concentration in respect of industry, but the process is logically inevitable, and, as we see, is proceeding with ever-increasing rapidity; and we may fairly conclude that the profit-making system as a whole, and as now operated, is inherently centralising in character.

With this concentration of financial power and consequent control, however, there is proceeding in industry another development, apparently contradictory in its results, but of the greatest importance in the consideration of the subject as a whole. During the period of transition between individual ownership and company or trust management, and under the stress of competition for markets, it became of vital importance to cut down the selling price of commodities, not so much intrinsically as in comparison with competitors; and as a means to this end, standardisation and quantity-production in large factories are of the utmost importance, carrying with them specialisation of processes, the substitution, wherever possible, of automatic and semi-automatic machinery for skilled workmanship, and the incorporation of the worker into a machine-like system of which every part is expected to function as systematically as a detail of the machine which he may operate. The objective has, to a considerable extent, been attained—the scientific management systems in factories (an outstanding instance of this policy) based on the researches of efficiency engineers such as Mr. F. W. Taylor and Mr. Frank Gilbreth, have resulted in a rate of production per unit of labour, hundreds or even thousands percent, higher than existed before their introduction.

As a bait for the worker these methods have commonly been accompanied by systems of payment-by-results, such as the premium-bonus system in its various forms as adapted by Halsey, Rowan, Weir, etc., round which has raged fierce controversy since in the very nature of things, being based on the consideration of profit, they were unable to take into account the operation of broad economic principles. It is no part of the argument with which we are concerned to discuss such systems in detail, but any unprejudiced and sufficiently technical consideration of them will carry the conviction that while the immediate effect of their introduction was undoubtedly to raise earnings and so apparently to delay the concentration of wealth, it was correctly recognised by the worker that his real wage tended to bear much the same ratio, or even to fall, in comparison with the cost of living, since the purchasing power of money in terms of food, clothes, and housing fell faster than his wages rose.

As the mechanical efficiency of production rose, therefore, discontent and industrial strife became accentuated, and an unstable equilibrium was only maintained by the operation of such factors as have become known under the names of “ca’canny,” restriction of output, etc., and before the war the operation of piece-work systems in large industrial engineering works almost invariably resulted in the establishment of a local ratio between time rates and piece-work earnings, generally ranging between 1.25 and 1.5 to 1. It is not necessary to discuss the ethics of such an arrangement; it is merely necessary to note that the settled policy of Labour, acting presumably on the best advice it could get in its own interests, \emph{was to exercise a control over production by fixing its own standard of output irrespective of time}. The situation created by the demand for munitions of all kinds during the war has, of course, profoundly modified this attitude, with the result that a temporary very large increase in real earnings undoubtedly took place in 1915 and 1916, taking the form of a rapid distribution of stored commodities; but it is quite questionable whether this level is even approximately maintained, and with the cessation of the wholesale sabotage of war, it will unquestionably fall as economic distribution through the wages system becomes ineffective; apart from actual scarcity.

Quite apart, therefore, from all questions of payment, there has grown up a spirit of revolt against a life spent in the performance of one mechanical operation devoid of interest, requiring little skill, and having few prospects of advancement other than by the problematical acquisition of sufficient money to escape from it.

The very efficiency with which factory operations have been sectionalised has resulted in a complete divorcing between the worker and the finished product, which is in itself conducive to the feeling that he is part of a machine in the final output of which he is not interested. His foreman and departmental heads are, from the largeness of the undertakings, almost inevitably out of human touch with him, while all the well-known phenomena of bureaucratic methods contribute to maintain a constant state of irritation and dissatisfaction; and in all these things is the nucleus of a centrifugal movement of formidable force. Nor is this feature confined to industrial life.. The connection between militarism and capitalism as vehicles for the expression of the will-to-power has frequently been * pointed out. By the device of universal liability to military service a general threat has been made operative which would appear, ultima ratio regis, to set the seal on the ability of authority to dictate the terms on which the existence of the individual can continue. But it is doubtful whether there ever was a time when this threat was held more lightly, and the disregard of consequences so widespread. It is not suggested that conscription either military or industrial is regarded with complacency; the exact opposite is, of course, the truth. But just for the reason that the whole conception of a militarist world is instinctively recognised as an anachronism, so, just to that extent, is the determination to defeat at any cost schemes involving compulsion, strengthened in the minds of a population normally acquiescent.

\chapter{Chapter Four}
\label{chapter-4}
We are, therefore, faced with an apparent dilemma, a world-wide movement towards centralised control, backed by strong arguments as to the increased efficiency and consequent economic necessity of organisation of this character (and these arguments receive support from quarters as widely separated as, say, Lord Milner and Mr. Sidney Webb), and, on the other hand, a deepening distrust of such measures bred by personal experience and observation of their effect on the individual. A powerful minority of the community, determined to maintain its position relative to the majority, assures the world that there is no alternative between a pyramid of power based on toil of ever-increasing monotony, and some form of famine and disaster; while a growing and ever more dissatisfied majority strives to throw off the hypnotic influence of training and to grapple with the fallacy which it feels must exist somewhere.

Now let it be said at once that there is no evasion of this dilemma possible by the introduction of questions of personality—a bad system is still a bad system no matter what changes are made in personnel. The power of personality is susceptible of the same definition as any other form of power, it is the rate of doing work; and the rate at which a given personality can change an organisation depends on two things; the magnitude of the change desired, and the size of the organisation. As it is hoped to make clear, the effect of a single organisation of this pyramidal character applied to the complex purpose of civilisation produces a definite type of individual, of which the Prussian is one instance. Pyramidal organisation is a structure designed to concentrate power, and success in such an organisation sooner or later becomes a question of the subordination of all other considerations to its attainment and retention. For this reason the very qualities which make for personal success in central control are those which make it most unlikely that success and the attainment of a position of authority will result in any strong effort to change the operations of the organisation in any external interest, and the progress to power of an individual under such conditions must result either in a complete acceptance of the situation as he finds it, or a conscious or unconscious sycophancy quite deadly to the preservation of any originality of thought and action.

It cannot be too heavily stressed at this time that similar forms of organisation, no matter how dissimilar their name, favour the emergence of like characteristics, quite irrespective of the ideals of the founders, and it is to the principles underlying the design of the structure, and not to its name or the personalities originally operating it, that we may look for information on its eventual performance.

In considering the objectionable features which have arisen from modern industrial and political systems in the light of this centralising tendency, it is instructive to turn for a moment to the examination of the differences which have developed in them with respect to those they have displaced, and without covering afresh the ground which has been sufficiently well traversed by the exponents of National Guilds, Syndicalism and other systems of industrial self-government, it may be well to point out that the industrial revolution of the late eighteenth and* early nineteenth centuries was largely marked in principle by the separation of the workman from the ownership of his tools and the control of his business policy.

All craft was handicraft; the equipment of a tradesman was of the simplest; the selling price of the product was practically material cost plus direct labour cost; direct labour cost was indistinguishable from profit, and practically the whole of it was available for the purchase of further material, and the product of other men’s industry. So far as our knowledge goes, and the theory of industry would confirm such an assumption, there was within the craft guilds no involuntary poverty or unemployment at all comparable to that with which we are too familiar, and, at any rate, within the circle of their influence the standard of material comfort rose directly in proportion to the total production, while at the same time the craftsman maintained a pride in his work and considerable independence.

With the advent of machinery came the intervention of the financier into industry; willing to provide the able craftsman with the means to extend the exercise of his skill on payment for his services. The development from this stage, though the small workshop run on borrowed money by the enterprising man who both worked himself and directed the work of others, to the larger factory in which the function of the craftsman ceased to be exercised by the employer, who retained only the direction and management; to the large limited liability company or Trust, in which the craftsman, the management, and the direction of policy, became still further separated, has been logical and rapid, and this development carries with it changes of a fundamental character.

Behind all effort lies the active or passive acquiescence of the human will, and this can only be obtained by the provision of an objective. By the separation of large classes into mere agents of a function, it has been possible to obtain the more or less complete cooperation of large numbers of individuals in aims of which they were completely ignorant, and of which had they been able to appreciate them in their entirety, they would have completely disapproved, while at the same time Education and Ecclesiasticism have combined to foster the idea, that so long as the orders of a superior were obeyed, no responsibility rested on the individual.

It is not, of course, suggested that commercial policy has been deliberately and uniformly dictated by unworthy motives—far from it; nor is it unlikely that had the processes of production and distribution been separated from any control over individual activity along other lines, its development might have been in the best interests of the community; but since it has been accompanied by a growing subjection of the individual to the machine of industrialism, it is quite unquestionable that the whole process of centralising power and policy and alleged responsibility in the brains of a few men whose deliberations are not open to discussion; whose interests, largely financial, are quite clearly in many respects opposed to those of the individuals they control, and whose critics can be victimised; is without a single redeeming feature, and is rendered inherently vicious by the conditions which operate during the selective process. When it is further considered that these positions of power fall to men whose very habit of mind, however kindly and broad in view it may be and often is in other directions, must quite inevitably force them to consider the individual as mere material for a policy—cannon-fodder whether of politics or industry—the gravity of the issue should be apparent.

Along with this development has gone a parallel change in the status of the individual. The apprentice, the journeyman and the master were all of one social class; the apprentice or journeyman dined at his master’s table and married his own or some other master’s daughter; the standard of life therefore without, of course, being identical, was comparable as between various grades. The implication of this was considerable—it involved a common standard to which everyday difficulties could be referred. A consideration of these facts, and a comparison of the conditions produced by them with those existing in our industrial districts in more recent years, has led reformers of the type of William Morris and John Ruskin to idealise this period and to place to the debit of machinery and quantity-production all the miseries and ugliness visible in the Midlands and the manufacturing North. This attitude seems mistaken, and here again we are met by a confusion between cause and effect: there is absolutely no virtue in taking ten hours to produce by hand a necessary which a machine will produce in ten seconds, thereby releasing a human being to that extent for other aims, but it is essential that the individual \emph{should he released}; that freedom for other pursuits than the mere maintenance of life should thereby be achieved.

How, then, are we to deal with this dilemma? It cannot seriously be contended that the advancement gained as a result of the application of material science to the requirements of society should be abandoned, and that men should abjure the use of anything more complicated than a hammer and chisel or a spinning wheel. But while progress in the replacement of manual effort by machinery seems both natural and beneficial, it is equally clear that the spiritual and intellectual revolt against the conditions which have grown up alongside this material progress is fundamental and widespread, and will not be satisfied by any mere betterment movement. The whole policy of Governments and industrialists alike in respect of this conflict of interest has been one of grudging compromise, partly as the result of the natural tendency of humanity to “laissez faire” methods and partly no doubt from a settled conviction that nothing but compromise was possible; that the existing order is based on natural law, and is not amenable to any radical modification, and that all critics are either cranks and dreamers, or else are solely actuated by a desire for the sweets of office. It is most important to recognise that there are two distinct problems involved in this dilemma: one technical, the other psychological, and it is just because the psychological aspect of industry has been confused with and subordinated to the technical aspect that we are confronted with so grave a situation at this time. There is little reason to doubt that we are rapidly attaining command of the means for the solution of any reasonable requirement of a purely technical nature, and it may be well therefore to consider briefly the usual methods which the modern industrial system has developed to deal with the organisation of large numbers of individuals to the end that their combined effort may result in commercial success.

Very broadly the main difference lies between what may be defined as the military and the functional systems of control, or some combination of the two, and these involve an interesting difference of conception.

As we have seen, the development of industrial activity has been very largely a practical application of the economic proposition in regard to the division of labour; the “military” organisation conceives a large business or a Government Department as an aggregation of human units to carry out on a large scale that which one immensely able and versatile man could do on a small scale, and, broadly considered, the perfect organisation of this character would be derived by dissecting the various attributes of the perfect one-man business, making each of them a Department, and staffing them with men who, in the aggregate, represented nothing but an expansion of that attribute. Fortunately, the perfect organisation of this character has yet to appear, but the effect of the endeavour to achieve it has quite definitely left its mark on civilisation—it is easy to distinguish the soldier and the civil servant, or even the infantryman and the bombardier, and the development due to the unbalanced exercise of one set only of perhaps many abilities resident in the human unit, is a very definite factor in the existing discontent and one which, if perpetuated, could only be increased by wider education.

A little consideration will at once suggest that this type of organisation carried out to its furthest limits is pyramid control in its simplest form, and it is clear that successive grades or ranks decreasing regularly in the number of units composing each grade, until supreme power and composite function is reached and concentrated at the apex, are definitely characteristic of it.

The next step is to split the functions of the higher ranks so that each unit therein becomes the head of a separate little pyramid, each of which as a whole furnishes the unit composing a larger pyramid; in every case, however, eventually concentralising power and responsibility in one man, representing the power of finance and of control over the necessaries of life.

Several points are to be noticed in the conditions produced by such an arrangement: Firstly, there is fundamental inequality of opportunity. The more any organisation, whether of society as a whole or any of the various aspects of it, approaches this form the more certain is it that there cannot possibly be any relation between merit and reward—it is, for instance, absurd to assume that there is only one possible head, for each railway company. Government Department, or great industrial undertaking. There is no doubt whatever that the intrigue which is a commonplace in such undertakings has its roots almost entirely in this cause, and contributes in no small degree to their notorious inefficiency.

Another objection which becomes increasingly important as the concentration proceeds is the divorce between power and detail knowledge. This difficulty is recognised in the appointment of official and unofficial intelligence departments which, of course, are in themselves the source of further abuses.

Having these points to some extent in mind, American industry has developed what is most unquestionably a very important modification of principle—that of functional control in place of individual control; that is to say, the individual is only controlled from one source in regard to one function—say time-keeping. In respect of such matters as technical methods he deals with an entirely different authority, and with still another in respect of pay.

The real objection to this is the effect on the source of specialised authority of so narrow a function as is demanded by much so-called scientific management, but there is very little doubt that the underlying idea does contain the germ of an industrial system which would be in the highest degree efficient if its psychological difficulties could be removed, and it is significant that this form of organisation produces its own type of personality.

It will be seen, therefore, that we have in the industrial field a double problem to solve: while retaining the benefits of mechanism for productive purposes, to obtain effective distribution of the results and to restore personal initiative.

The proposition which is being urged from orthodox capitalistic quarters as a means of dealing with this situation is a little ingenuous. It consists of an intensification policy by which, in some mysterious way, all the unpleasant features, by being exaggerated, are to disappear, and it is usually summed up at the moment in the phrase, “We must produce more.” A fair statement of this demand for unlimited and intensified manufacturing would no doubt be something after this fashion:-—

\begin{enumerate}
	\item We must pay for the war and for betterment schemes.


	\item This means high taxes.


	\item Taxes must come from profits and earnings, which are parts of one whole.


	\item High earnings, high profits, and low labour costs, and low selling and competitive costs, can only be combined if increased output is obtained.


	\item High earnings will mean wider markets.



\end{enumerate}
Now this is a very specious argument; a large number of people, whose instincts warn them that there is a fallacy somewhere, have not felt themselves able to offer any effective criticism of it, since some practical knowledge of technique is involved. The labour attitude has either been a simple non-possumus, or a re-statement of the evils of capitalistic profit-making, together with sufficiently pungent inquiry into the qualifications of the holders of the major portion of the securities representing Government indebtedness, and their title to rank as the winners of the war, and the chief beneficiaries of the peace. All this is quite to the point, but it is not even the chief economic objection to such a policy.

First of all, let it be admitted that a considerable amount of manufacturing will have to be done, firstly, to reinstate the devastated areas, and afterwards to meet the accumulated demand, and these together will provide an outlet for a very large quantity of manufactured goods. These goods will not, of course, be furnished for nothing, and the money to pay for them will in the main be supplied by loans, which to begin with, clearly mean more taxes for someone where the work done is on public account. But, says the super-producer, this money will be distributed in wages, salaries and profits, which will enable the whole population (at any rate of this country, where we propose to do our manufacturing so long as labour and other conditions are favourable) to buy more goods, or, conversely, save more money, and eventually enjoy more leisure and freedom.

Let us give to this statement the attention it deserves, because on it hangs the fate of a whole economic system. If it is true as it stands, then the whole system which stands behind it, the fight for markets, the cartels, trusts, and combines, and the other machinery of competitive trade, are justified at any rate by national self-interest. In order then to make this analysis it is unavoidable that we should enter into some detail with regard to the accountancy of manufacturing; not forgetting that the unequal distribution of wealth is an initial restriction on the free sale of commodities, and that in consequence what we are aiming at in order to meet the final contention of the argument, is not an expansion of figures, but an equalisation of real purchasing power.

Now, purchasing power is the amount of goods \emph{of the description desired} which can be bought with the sum of money available, and it is clearly a function of price. It is a widely spread delusion that price is simply a question of supply and demand, whereas, of course, only the upper limit of price is thus governed, the lower limit, which under free competition would be the ruling limit, being fixed by cost plus the minimum profit which will provide a financial inducement to produce. It is important to bear this in mind, because it is frequently assumed that a mere glut of goods will bring down prices quite irrespective of any intrinsic economy involved in large scale production. Unless these goods are all absorbed, the result may be exactly opposite, since deterioration must go into succeeding costs. Cost is the accumulation of past spendings over an indefinite period, whereas cash price requires a purchasing power effective at the moment of purchase.

Where competition is restricted by Trusts, price is cost plus whatever profit the Trust considers it politic to charge.

\chapter{Chapter Five}
\label{chapter-5}
Looked at from this standpoint it is fairly clear that the kernel of the problem is factory cost, since it is quite possible to conceive of a limited company in which the shares were all held by the employees, either equally or in varying proportions, according to their grade, and the selling costs were internal—that is to say, all advertising was done by the firm itself, and the cost of its salesmen, etc., was either negligible, or confined to their salaries. We should then have the complete profit-sharing enterprise in its ultimate aspect, and the argument against Capitalism in its usual form would not arise.

Such an undertaking would, let us assume, make a complicated engineering product, requiring expensive plant and machinery, and would absorb considerable quantities of power and light, lubricants, etc., much of which would be wasted; and would inevitably produce a certain amount of scrap the value of which would be less than the material in the form in which it entered the works. The machinery would wear out, and would have to be replaced and maintained, and generally it is clear that for each unit of production there would be three main divisions of factory cost, the “staple” raw material, the wages and salaries, and a sum representing a proportion of the cost of upkeep on the whole of the plant, which might easily equal 200\%, of the wages and salaries. As the plant became more automatic by improvements in process, the ratio which these plant costs bore to the cost of labour and salaries would increase. The factory cost of the total production, therefore, would be the addition of these three items: staple material, labour and salaries, and plant cost, and with the addition of selling charges and profit, this would be the selling price.

As a result of the operations of the undertaking, the wealth of the world would thus be apparently increased by the difference between the value of all the material entering the factory, and the total sum represented by the selling price of the product. But it is clear that the total amount distributed in wages, salaries and profit or dividends, would be less by a considerable sum (representing purchases on factory account) than the total selling price of the product, and if this is true in one factory it must be true in all. Consequently, the total amount of money liberated by manufacturing processes of this nature is clearly less than the total selling price of the product. This difference is due to the fact that while the final price to the consumer of any manufactured article is steadily growing with the time required for manufacture, during the same time the money distributed by the manufacturing process is being returned to the capitalist through purchases for immediate consumption.

A concrete example will make this clear. A steel bolt and nut weighing ten pounds might require in the blank about eleven and a half pounds of material representing, say, 3s. 6d. The nett selling price of the scrap recovered would probably be about one penny. The wages value of the total man-hours expended on the conversion from the blank to the finished nut and bolt might be 5s., and the average plant charge 150\%, on the direct time charge, i.e.\textasciicircum{} 7s. 6d. The factory cost would, therefore, be 15s. lid., of which 7s. 6d., or just under one-half, would be plant charge. Of this plant charge probably 75\%., or about 5s. 7d., is represented by the sum of items which are either afterwards wiped off for depreciation and consequently not distributed at all at that time, or are distributed in payments outside the organisation, which payments clearly must be subsequent to any valuation of the articles for which they are paid, and so do not affect the argument. Without proceeding to add selling charges and profit it must be clear that a charge of 15s. lid. on the world’s purchasing power has been created, of which only 6s. 10d. is distributed in respect of the specific article under consideration, and that if the effective demand exists at all in a form suitable for the liquidation of this charge, it must reside in the banks.

But we know that the total increase in the \emph{personal} cash accounts in the banks in normal times is under 3\% of the wages, salaries and dividends distributed, consequently it is not to these accounts that we must look for effective demand. There are two sources remaining; loan-credit, that is to say, purchasing power \emph{created} by the banks on principles which are directed solely to the production of a positive financial result; and foreign or export demand. Now loan-credit is never available to the consumer as such, because consumption as such has no commercial value. In consequence loan-credit has become the great stimulus either to manufacture or to any financial or commercial operation which will result in a profit, that is to say, an inflation of figures.

An additional factor also comes into play at this point. All large scale business is settled on a credit basis. In the case of commodities in general retail demand, the price tends to rise above the cost limit, because the sums distributed in advance of the completion of large works become effective in the retail market, while the large works, when completed, are paid for by an expansion ot credit. This process involves a continuous inflation of currency, a rise in prices, and a consequent dilution in purchasing power.

The reason that the decrease in the consumer’s purchasing power has not been so great as would be suggested by these considerations is, of course, largely due to intrinsic cheapening ot processes which would, if not defeated by this dilution of the consumer’s purchasing power, have brought down prices faster than they have risen.

There are thus two processes at work; an intrinsic cheapening of the product by better methods, and an artificial decrease in purchasing power due to what is in effect the charging of the cost of all waste and inefficiency to the consumer. And it is clear that under this system the greater the volume of production the larger will be the absolute value of the waste which the consumer has to pay for, whether he will or no, because as the bank credits are created at the instance of the manufacturer, and repaid out of prices, each article produced dilutes, by the ratio of its book price to all the credits outstanding, the absolute purchasing power of the money held by any individual.

These facts are quite unaffected by the perfectly sound argument that increased production means decreased cost per piece, since it is the total production price which has to be liquidated.

Already there is not very much left of the argument for the innate desirability of unlimited, unspecified and intensified manufacturing under the existing economic system, but more trouble yet is ahead of it. While the ratio of plant charges to total wages and salaries cost is less than 1:1 over the whole range of commodities, a general rise in direct rates of pay may mean a rise (but not a proportionate rise) in the purchasing power of those who obtain their remuneration in this way. But when by the increased application of mechanical methods the average overhead charge passes the ratio of one to one (which it rapidly will, and should do on this basis of calculation) every general increase in rates of pay of “direct” labour may mean an actual decrease in real pay, because the consumer is only interested in ultimate products and overhead charges do not represent ultimate products in existence.

The whole argument which represents a manufactured article as an access of wealth to the country and to everyone concerned, no matter what its description and utility\textasciicircum{} so long as by any method it can be sold and wages distributed in respect of it, will, therefore, be seen to be a dangerous fallacy based on an entirely wrong conception,, which is epitomised in the use of the word “production,” and fostered by ignorance of financial processes. Manufacturing of any kind whatever, even agriculture in a limited sense, is the conversion of one thing into another, which process is only advantageous to the extent that it subserves a definite requirement of human evolution. In any case, it shares with all other conversions the characteristic of having only a fractional efficiency, and the waste of effort involved, although being continually reduced by improvements of method, still can only be paid for in one way, by effort on the part of somebody.

If this effort is useful effort–“useful” in the sense that a definite, healthy and sane human requirement is served—the wealth and standard of living of the community may thereby be enhanced. If the effort is aimless or destructive, the money attached to it does not alter the result.

The financial process just discussed therefore clearly attaches a concrete money value to an abstract quality not proven, and as this money value must be represented somewhere by equivalent purchasing power in the broadest sense, misdirected effort which appears in cost forms a continuous and increasing diluent to the purchasing value of effort in general.

Now it has already been emphasised that, at the moment, economic questions are of paramount importance, because the economic system is the great weapon of the will-to-power. It will be obvious that if the economic problem could be reduced to a position of minor importance—in other words, if the productive power of machinery could be made effective in reducing to a very small fraction of the total man-hours available, the man-hours required for adapting the world’s natural resources to the highest requirements of humanity—the “deflation” of the problem would, to a very considerable extent, be accomplished. The technical means are to our hands; the good will is by no means lacking and the opportunity is now with us. But it should be clearly recognised that waste is not less waste because a money value is attached to it, and that the machinery of remuneration must be modified profoundly since the sum of the wages, salaries and dividends, distributed in respect of the world’s production will buy an ever-decreasing fraction of it.

It is one of the most curious phenomena of the existing economic system that a large portion of the world’s energy, both intellectual and physical, is directed to the artificial stimulation of the desire for luxuries by advertisement and otherwise, in order that the remainder may be absorbed in what is frequently toilsome, disagreeable and brutalising work; to the end that a device for the distribution of purchasing power may be maintained in existence. The irony of the situation is the greater since the perfecting of the organisation to carry on this vicious circle, carries with it as we have just seen, a complete negation of all real progress.

The common factor of the whole situation lies in the simple facts that at any given period the material requirements of the individual are quite definitely limited—that any attempt to expand them artificially is an interference with the plain trend of evolution, which is to subordinate material to mental and psychological necessity; and that the impulse behind unbridled industrialism is not progressive but reactionary, because its objective is an obsolete financial control which forms one of the most effective instruments of the will-to-power, whereas the correct objectives of industry are two-fold; the removal of material limitations, and the satisfaction of the creative impulse.

It is for this reason that while, as we see, the effect of the concrete sum distributed as profit is over-rated in the attacks made on the Capitalistic system, and is of small and diminishing importance as compared with the delusive accounting system which accompanies it, and which acts to reduce consistently the purchasing power of effort, it is, nevertheless, of prime importance as furnishing the immediate “inducement to produce,” which is a false inducement in that it claims as “wealth” what may just as probably be waste.

If by wealth we mean the original meaning attached to the word: \emph{i.e.}, " well-being," the value in well-being to be attached to production depends entirely on its use for the promotion of well-being (unless a case is made out for the moral value of factory life), and bears no relation whatever to the value obtained by cost accounting.

Further, if the interaction between production for profit and the creation of credit by the finance and banking houses is understood, it will be seen that the root of the evil accruing from the system is in the constant filching of purchasing power from the individual in favour of the financier, rather than in the mere profit itself.

Having in view the importance of the issues involved, it may be desirable to summarise the conclusions to be derived from a study of the methods by which the price of production is based on cost under the existing economic arrangements. They are as follows:-—

\begin{enumerate}
	\item Price cannot normally be less than cost plus profit.


	\item Cost includes all expenditure on product.


	\item Therefore, cost involves all expenditure on consumption (food, clothes, housing, etc.), paid for out of wages, salary or dividends as well as all expenditure on factory account, also representing previous consumption.


	\item Since it includes this expenditure, the portion of the cost represented by this expenditure has already been paid by the recipients of wages, salaries and dividends.


	\item These represent the community; therefore, the only distribution of real purchasing power in respect of production over a unit period of time is the surplus wages, salaries and dividends available after all subsistence, expenditure and cost of materials consumed has been deducted. The surplus production, however, includes all this expenditure in cost, and, consequently, in price.


	\item The only effective demand of the consumer, therefore, is a few percent, of the price value of commodities, and is cash credit. The remainder of the Home effective demand is loan credit, which is controlled by the banker, the financier, and the industrialist, in the interest of production with a financial objective, not in the interest of the ultimate consumer.



\end{enumerate}
It will be necessary to grasp the significance of these considerations, which can hardly be over-rated in its effect on the break-up of the existing economic system, in order to appreciate the result of a change in the control of credit and the method of price fixing, with which it is proposed to deal at a later stage.

\chapter{Chapter Six}
\label{chapter-6}
It will be readily understood that the difficulties which are seen to be inherent in the policy of super-production are only an accentuation of those with which we were only too familiar prior to the outbreak of war, and it may be contended and, in fact, it frequently is stated, that even with the unemployment statistics at their minimum point and the Nation at its maximum activity in Industry, there is still not enough product to go round. Recently, for instance. Professor Bowley has estimated that the total surplus income of the United Kingdom in excess of £160 per annum is only £250,000,000, which would mean, if distributed to 10,000,000 heads of families, £25 per annum per family, assuming that this distribution did not reduce the production of wealth.

The figures themselves have been criticised; but, in any case, the whole argument is completely fallacious, because it takes no account whatever of loan credit, which is by far the most important factor in the distribution of production, as we have already seen. What it \emph{does} show is that the purchasing power of effort is quite insignificant in comparison with its productive power.

But it may be advisable to glance at some of the proximate causes operating to reduce the return for effort; and to realise the origin of most of the specific instances, it must be borne in mind that the \emph{existing economic system distributes goods and services through the same agency which induces goods and services}, \emph{i.e.}, payment for work in progress. In other words, if production stops, distribution stops, and, as a consequence, a clear incentive exists to produce useless or superfluous articles in order that useful commodities already existing may be distributed.

This perfectly simple reason is the explanation of the increasing necessity of what has come to be called economic sabotage; the colossal waste of effort which goes on in every walk of life quite unobserved by the majority of people because they are so familiar with it; a waste which yet so over-taxed the ingenuity of society to extend it that the climax of war only occurred in the moment when a culminating exhibition of organised sabotage was necessary to preserve the system from spontaneous combustion.

The simplest form of this process is that of “making work”; the elaboration of every action in life so as to involve the maximum quantity and the minimum efficiency in human effort. The much-maligned household plumber who evolves an elaborate organisation and etiquette probably requiring two assistants and half a day, in order to “wipe” a damaged water pipe, which could, by methods with which he is perfectly familiar, be satisfactorily repaired by a boy in one-third the time; the machinist insisting on a lengthy apprenticeship to an unskilled process of industry, such as the operation of an automatic machine tool, are simple instances of this. A little higher up the scale of complexity comes the manufacturer who produces a new model of his particular speciality, with the object, express or subconscious, of rendering the old model obsolete before it is worn out. We then begin to touch the immense region of artificial demand created by advertisement; a demand, in many cases, as purely hypnotic in origin as the request of the mesmerised subject for a draught of kerosene. All these are instances which could be multiplied and elaborated to any extent necessary to prove the point.

In another class comes the stupendous waste of effort involved in the intricacies of finance and book-keeping; much of which, although necessary to the competitive system, is quite useless in increasing the amenities of life; there is the burden of armaments and the waste of materials and equipment involved in them even in peace time; the ever-growing bureaucracy largely concerned in elaborating safeguards for a radically defective social system; and, finally, but by no means least, the cumulative export of the product of labour, largely and increasingly paid for by the raw material which forms the vehicle for the export of further labour.

All these and many other forms of avoidable waste take their rise in the obsession of wealth defined in terms of money; an obsession which even the steady fall in the purchasing power of the unit of currency seems powerless to dispel; an obsession which obscures the whole object and meaning of scientific progress and places the worker and the honest man in a permanently disadvantageous position in comparison with the financier and the rogue. It is probable that the device of money is a necessary device in our present civilisation; but the establishment of a stable ratio between the use value of effort and its money value is a problem which demands a very early solution, and must clearly result in the abolition of any incentive to the capitalisation of any form of waste.

The tawdry “ornament,” the jerry-built house, the slow and uncomfortable train service, the unwholesome sweetmeat, are the direct and logical consummation of an economic system which rewards variety, quite irrespective of quality, and proclaims in the clearest possible manner that it is much better to “do” your neighbour than to do sound and lasting work.

The capitalistic wage system based on the current methods of finance, so far from offering maximum distribution, is decreasingly capable of meeting any requirement of society fully. Its very existence depends on a constant increase in the variety of product, the stimulation of desire, and in keeping the articles desired in short supply.

\chapter{Chapter Seven}
\label{chapter-7}
If the preceding endeavour to marshal into some sort of coherent pattern the facts of the general economic and social situation as it exists at present has been to any extent successful, it will be evident that the real antagonism which is at the root of the upheaval with which we are faced is one which appears under different forms in every aspect of human life. It is the age-long struggle between freedom and authority, between external compulsion and internal initiative, in which all the command of resources, information, religious dogma, educational system, political opportunity and even, apparently, economic necessity, is ranged on the side of authority; and ultimate authority is now exercised through finance. This antagonism does, however, appear at the present time to have reached a stage in which a definite victory for one side, or the other is inevitable—it seems perfectly certain that either a pyramidal organisation, having at its apex supreme power, and at its base complete subjection, will crystallise out of the centralising process which is evident in the realms of finance and industry, equally with that of politics, or else a more complete decentralisation of initiative than this civilisation has ever known will be substituted for external authority. The issue transcends in importance all others: the development of the human race will be radically different as it is decided one way or another, but as far as it is possible to judge, the general advantage of the individual will lie with the retention of a measure of coordination in all mechanical organisation, combined with the evolution of progressively decentralised initiative, largely by the displacement of the power of centralised finance.

The implication of this is a challenge, which will become more definite as time goes on, to external authority as to its right to adjudicate on the absolute value, expressed in terms of commodities, of various forms of activity. Even now, the practical difficulty of estimating the relation between material reward and individual effort is becoming almost insuperable, even in the cases where an honest effort is made to arrive at some solution. The various movements for the grant of a minimum living wage, the demand for the recognition of the “right to work” (\emph{i.e.}, to draw pay) are all symptoms of the breakdown of the financial “law” of supply and demand in its application to economic problems.

Still another significant feature of the inadequacy of the economic structure is the increase of voluntary unpaid effort and the large amount of energy devoted to games. There is absolutely no concrete difference between work and play unless it be in favour of the former—no one would contend that it is inherently more interesting or pleasurable, to endeavour to place a small ball in an inadequate hole with inappropriate instruments, than to assist in the construction of a Quebec Bridge, or the harnessing of Niagara. But for one object men will travel long distances at their own expense, while for the other they require payment and considerable incentive to remain at work.

The whole difference is, of course, psychological; in the one case there is absolute freedom of choice, not of conditions, but as to whether those conditions are acceptable; there is some voice in control, and there is an avoidance of monotony by the comparatively short period of the game, followed by occupation of an entirely different order. But the efficiency of the performance as compared with the efficiency of the average factory worker is simply incomparable—any factory which could induce for six months the united and enthusiastic concentration of, say, an amateur football team would produce quite astonishing results.

Now, it may be emphasised here at once, that there is absolutely no future for inefficiency as a cult; the whole promise of a brighter, probably a very bright, future for the world lies in doing the best possible things in the best possible way. In industrial affairs the principle of the maximum efficiency of effort per unit of time is so patently unassailable that its enunciation would hardly be necessary, but that the proposition carries with it a very different conception of efficiency than the narrow “business” meaning commonly attached to the word, and in consequence it is the fashion amongst the less progressive elements of society to attack any demand for improved conditions as simply an attempt to substitute sloth and incapacity for energy and capability. While, therefore, a readjustment of system and, above all, a complete reconsideration of objective is necessary, it is probable that \emph{the basis of such changes must be economic, with political and financial systems auxiliary rather than definitive}, and it is certain that a revision of economic policy, to be stable, must result in higher economic efficiency, even though the very aim of that higher efficiency is to reduce economic problems to a very subordinate position. And the higher psychological efficiency of voluntary effort is clearly a step to this end.

We have just seen that merely increased production under existing conditions will not achieve any economic stability because there are at least two quite irreconcilable criteria governing the scope of the operations proposed. There is, on the one hand, the adjustment of manufacturing of all sorts to the opportunity of sale (not by any means always profitable sale) and this is a purely artificial and yet all-powerful consideration under present financial systems, and constitutes the effective demand.

And there is, on the other hand, the growing \emph{real} demand, first for food, clothing and shelter and then for participation in the wider life which modern progress has made possible, such demand being quite irrespective of capacity to pay in money. And the reconciliation of these two interests means the defeat of the will-to-power by the will-to-freedom, and in this reconciliation is involved a modification of economic distribution.

Now if there is any sanity left in the world at all, it should be obvious that the real demand is the proper objective of production, and that it must be met from the bottom upwards, that is to say, there must be first a production of necessaries sufficient to meet universal requirements; and, secondly, an economic system must be devised to ensure their practically automatic and universal distribution; this having been achieved it may be followed to whatever extent may prove desirable by the manufacture of articles having a more limited range of usefulness. All financial questions are quite beside the point; if finance cannot meet this simple proposition then finance fails, and will be replaced. It has been estimated that two hours per week of the time of every fit adult between the ages of 18 and 45 would provide for a uniformly high standard of physical welfare under existing conditions, and without endorsing the exact figures it is perfectly certain that distribution and not manufacture is the real economic problem and is at present quite intolerably unsatisfactory. There is no need to assume that the whole machinery of business as we know it must be scrapped; in fact, the machinery of business, as machinery, is highly efficient; but it must undoubtedly be adjusted so that no selfish desire for domination can make it possible for any interest to hold up distribution on purely artificial grounds. Since the analysis of existing conditions which we have undertaken shows that any centralised administrative organisation is certain to be captured by some interest antagonistic to the individual, it seems evident that it is in the direction of decentralisation of control that we must look for such alteration in the social structure as would be self-protective against capture for interested purposes.

As we have already seen, alongside the concentration of political and industrial power a powerful decentralising force is already beginning to show itself in various forms. In considering the manifestation of this force it will be observed that at the moment it is seeking expression through organisation—in new forms, but for the present operating with old sources of energy, chiefly negative in character, such as the strike. To be effective, however, against positive centralisation, positive decentralisation will have to come—decentralised economic power is necessary.

Among the more important of these forms is the shop steward or rank-and-file movement in industry, and the workmen’s councils in politics, both purely decentralising in tendency, quite apart from any special policy for the furtherance of which they may be used. The apprehension with which the movements are regarded by the reactionary capitalist is based far more on a recognition of the difficulties such a scheme of organisation offers to successful corruption and capture than to any regard for the specific items in the policy it may for the moment represent; most of which have been previously parried with ease when presented through delegated Trade Union leaders, whose position of authority have been perforce achieved by exactly the methods best understood by those with whom they have to deal.

As the Shop Steward movement is the most definite industrial recognition from the Labour side, of the necessity for decentralisation, some examination of the general scheme is of interest. The actual details of the organisation vary from place to place, trade to trade, and even day to day; but the essence of the idea consists in the adoption of a decentralised unit of production such as the “shop” or department, and the substitution of actual workers in considerable numbers, for the paid Trade Union official as the nucleoli of both industrial and political power (although the political power is not exercised through Parliamentary channels).

The shop steward is generally “industrial” rather than “craft” in interest; that is to say, he represents a body of men who produce an article, rather than a section who perform one class of operation for widely different ends; but there is nothing inherently antagonistic as between the two conceptions of function, Industrial Unionism being largely a militant device. He is quite limited in his sphere of action, but initiates general discussion on the basis of first-hand information, and forms a link between the decentralised industrial unit and other units which may be concerned. The practical effect of the arrangement is that the spokesmen are never out of touch with those for whom they speak, since the normal occupation and remuneration of representatives is similar to that of those they represent; and should any cleavage occur, a change of representative can be easily secured. The official concerned has, in theory, no executive authority whatever, nor can he take any action not supported by his co-workers, \emph{i.e.}, the direction of policy is from the bottom upwards instead of the top downwards. The individual shop stewards are banded together in a shop stewards’ committee, which has again only just as much authority as the individual workers care to delegate to it.

It is, of course, obvious that the permanent success of any arrangement of this character depends on a common recognition amongst the individuals affected by the organization, of certain principles as “confirming standards of reference.” In short, it would be impossible to administer a complicated manufacturing concern on any such principles unless the general body of employees had a general appreciation of the fundamental necessities of the business, inclusive of direction and technical design.

In other words, and in a more general sense all \emph{political} arrangements of this or any other description simply provide a mechanism for the administration of an agreed system—they are not, and cannot in their very nature be that system in itself.

Where, of course, it is clear that there is a confusion of function is, that the shop steward claims control not only of the conditions of production, but eventually of the terms of distribution. This confusion is quite inevitable at present, but is not necessarily permanent, and is obviously undesirable. It is based on the fallacy that labour, as such, produces all wealth, whereas the simple fact is that production is 95\%, a matter of tools and process, which tools and process form the cultural inheritance of the community not as workers, but as a community, and as such the community is most clearly the proper though far from being the legal, administrator of it.

\chapter{Chapter Eight}
\label{chapter-8}
Admitting, then, that any decentralised scheme of society must first justify itself economically, it is necessary to grapple with, at any rate, the main features of the radical reconstruction necessary before any attempt can be made to forecast the political aspect.

The starting point is clearly a reasonably uniform and plentiful distribution of simple necessaries; food, clothes, housing, etc.

Now the actual production of these articles presents no difficulties whatever. Notwithstanding the diversion of the major portion of the world’s energy for four years to purposes of destruction, the actual economic want in the world has been almost entirely artificial, \emph{i.e.}, has been confined either to countries effectively blockaded, or else lacking the mechanical facilities for effective distribution. In fact, it is most significant that while useful (in a peace sense) production has been enormously reduced in Great Britain during the war, the standard of comfort has been more uniformly high than ever before.

The explanation of this is simple: The payments made in wages have increased, prices and the production of luxuries have been partly controlled, and sabotage has disposed of useless product, and so kept up wage distribution.

The practical problem, then, is to make it certain that commodities are produced under satisfactory conditions, and equally certain that they are distributed according to necessity, and the organisation for these purposes may well determine the social structure, inasmuch as a complete success would be the most powerful incentive to the adoption of similar methods in less fundamental directions.

Profiting by the deduction made from the examination already made of the results of various types of organisation, it may be repeated that the best results would seem probable from a coordinated organisation for purposes of technique with the greatest decentralisation of initiative in the use of the facilities so provided.

Now it should be clearly grasped at the outset that at least two main problems are involved in the question at issue, which may be broadly defined as that of the producer and the consumer; and not only are these entirely separate, but, rightly considered, they are on completely different planes of existence.

The problem of the consumer is essentially material; he is concerned with quality, variety, price, supply; he is concerned with \emph{product}.

On the contrary, the producer is almost entirely concerned with psychological issues; fatigue, interest, welfare, hours of labour, all of which, \emph{qua} producer pure and simple, are broadly summed up in the word “contentment.”

The consumer is interested in distribution; the producer is concerned with effort. While the producer and the consumer are frequently combined in the same person, a recognition of these distinctions will make it easier to define the powers which should belong to each.

It is particularly necessary to emphasise this distinction since the existing structure of industry based on finance takes it for granted that the possession of large quantities of goods, or their equivalent purchasing power in money, is a good and sufficient reason for the exercise of a preponderating voice in the conditions and processes of production.

We say, and it is only now that it is faintly contested, that he who pays the piper calls the tune. The idea that it is the hearer who is primarily concerned in the tune, the piper primarily in the instrument, and the payment a mere convenience as between the two parties, is so novel to large numbers of unthinking persons, that it is only natural to expect violent opposition to the world-wide efforts being made to reconstitute society on these very principles.

Bearing these distinctions in mind it will be recognised that there are two separate lines along which to attack the situation presented by the dissatisfaction of the worker with his conditions of work, and the not less serious discontent of the consumer with the machinery of distribution; and these may be called mediaevalism and ultra-modernism.

Mediaevalism seems to claim that all mechanical progress is unsound and inherently delusive; that mankind is by his very constitution compelled, under penalty of decadence, to support himself by unaided skill of hand and eye. In support of its contentions it points to the Golden Age of the fourteenth century in England, for example, when real want was comparatively unknown, and green woods stood and clear rivers ran where the slag-heaps and chemical works of Widnes or Wednesbury now offend the eye and pollute the air. When arts and crafts made industry almost a sacrament, and faulty execution a social and even a legal offence; when the medium of exchange was the Just Price, and the idea of buying in the cheapest and selling in the dearest market, if it existed, was classed with usury and punished by heavy penalties.

While appreciating the temptation to compare the two periods to the very great disadvantage of the present, it does not seem possible to agree with the conclusion of the Mediaevalist that we are in a cul-de-sac from which the only exit is backwards; and it is proposed to make an endeavour to show that there is a way through, and that we may in time regain the best of the advantages on which the Mediaevalist rightly sets such store, retaining in addition a command over environment, which he would be the first to recognise as a real advance; a solution which may be described as Ultra-Modernist.

In order to do this, certain somewhat abstract assumptions are necessary, and it has been the object of the preceding pages to present as far as possible the data on which these assumptions are made. They are as follows:-—

\begin{enumerate}
	\item The existing difficulties are the immediate result of a social structure framed to concentrate personal power over other persons, a structure which must take the form of a pyramid. Economics is the material key to this modern riddle of the sphinx because power over food, clothes, and housing is ultimately power over life.


	\item So long as the structure of Society persists personality simply reacts against it. Personality has nothing to do with the effect of the structure; it merely governs the response of the individual to conditions he cannot control except by altering the structure.


	\item It follows that general improvement of \emph{conditions} based on personality is a confusion of ideas. Changed personality will only become \emph{effective} through changed social structure.


	\item The pyramidal structure of Society gives environment the maximum control over individuality. The correct objective of any change is to give individuality maximum control over environment.



\end{enumerate}
If these premises are accepted it seems clear that the first and probably most important step is to give the individual control of the necessaries of life on the cheapest terms possible. What are these terms? What is the fundamental currency in which the individual does in the last analysis liquidate his debts? A little consideration must make it clear that there can be only one reply; that the individual only possesses inalienable property of the one description; potential effort over a definite period of time. If this be admitted, and it is inconceivable that anyone would seriously deny it, it follows that the real unit of the world’s currency is effort into time—what we may call the time-energy unit.

Now, time is an easily measurable factor, and although we cannot measure human potential, because we have at present no standard, it is, nevertheless, true that for a given process the number of human time-energy units required for a given output is quite definite, and therefore, the cheapest terms on which the individual can liquidate his debt to nature in respect of food, clothes, and shelter, is clearly dependent on process; and by getting free of this debt with the minimum expenditure of time-energy units of which his individual supply varies, but is, nevertheless, quite definite at any given time, he clearly is so much the richer in the most real sense in that he can control the use to be made of his remaining stock.

But, and it is vital to the whole argument, improved process must be made the servant of this objective, that is to say, a process which is improved must, by the operation of a suitable economic system decrease the time-energy units demanded from the community, or to put the matter another way all improvements in process should be made to pay a dividend to the community. (It will be noted that an admission of the theorem is a complete condemnation of payment by results as commonly understood; that is to say, an arrangement of remuneration designed to foster an increasing use of time-energy units.) The primary necessaries of life as above defined, \emph{i.e.}, food, clothes and shelter, have an important characteristic which differentiates them from what we may call conveniences and luxuries; they are quite approximately constant in quantity per head of the population; in other words, the average human being requires as a groundwork for his daily life a definite number of heat units in the form of suitable food, a definite minimum quantity of clothing and a definite minimum space in which to sleep and work, and the variation between the minimum and the maximum quantity of each that he can utilise with advantage to himself is not, broadly speaking, very great.

This fact renders it perfectly feasible (it has already very largely been accomplished)–to estimate the absolute production of foodstuffs required by the world’s population; the time-energy units required at the present stage of mechanical and scientific development to produce those foodstuffs; and the time-energy units approximately available. Accuracy in these estimates is unnecessary, since there is not the very smallest doubt that the margins are so large that it is only the failure of “effective demand” under existing circumstances which has prevented over-production. The most superficial consideration of the earnings of agriculture before the war must make this obvious.

There is good ground for stating that the subsistence basis of the civilised world stated thus in time-energy units represents a few minutes’ work per day for all adults between the ages of 18 and 40.

Exactly the same principle is applicable to the provision of clothing and housing, and the “maintenance rate” in respect of these staple commodities as distinct from the “exploitation effort” necessary to put the world on a satisfactory basis does not again exceed a few minutes per day per head on the assumption that the fullest use is made of natural sources of energy, and that all the human effort specifically connected with the system of production for profit is eliminated. The exact figures are beside the point, but something over three hours’ work per head per day is ample for the purpose of meeting consumption and depreciation of all the factors of modern life under normal conditions and proper direction.

Now, such a line of policy is clearly based on coordination of design, but it evolves under certain conditions radical decentralisation of initiative.

These conditions are firstly definite productions of \emph{ultimate products} to a programme, and consequent limitation of output to that programme; and, secondly, the provision of an incentive to produce which shall ensure the distribution of the article produced. The basis of the first condition has just been indicated briefly; the provision of an incentive requires more extended analysis.

There is a disposition on the part of certain idealistic people, and, in particular, in quarters obsessed by the magic of the State idea, to decry the necessity of any organised incentive in industry at all. They seem to suggest either that the problem is merely one of designing a huge machine of such irresistible power that no incentive is necessary because no resistance is possible, or, alternatively, that the mere creative impulse ought to be sufficient to induce every individual to give of his best without any thought of personal benefit. In regard to the former idea, it may be said that quite apart from its fundamental objection it is quite impracticable; and in regard to the latter that it is not yet, nor for a very considerable time, likely to be practicable to satisfy the creative impulse through the same channels as those used for the economic business of the world.

Under existing conditions there is much necessary work to be done which cannot fail to be largely of a routine nature, and the provision of an incentive external to the performance of the immediate task seems both practically and morally sound.

First of all, some consideration of the defects of existing incentives is necessary in order to meet the difficulties so exposed.

Broadly, remuneration, or the system by which the amenities of civilisation are placed at the disposal of the individual, is of three varieties; payment by financial manipulation (profit), payment by time (salaries and time-rate wages), and payment by results (piecework in all its forms), and it should be noticed that only the first of these combines possession of the amenities with opportunities for their fullest use.

Payment by financial manipulation, whether through the agency of profit (other than that earned by personal endeavour), stock manipulation or otherwise, is quite definitely anti-social. It operates to neutralise all progress towards real efficiency by diluting the medium of exchange, and by this process it will quite certainly bring about the downfall of the social order to which it belongs, largely through the operation of the factory economic system already discussed.

Payment by time fails for two practical reasons; it is based on the operation of the fallacy that the \emph{value} of a thing bears any relation to the demand for it, and the assumption that money has a fixed value. Because of the first reason it clearly penalises genuine initiative (because there is no demand for the unknown), and because of the second, it fosters aggression. The policy of Trade Unions in regard to time rates of pay has simply been successful to the extent that it has used its organised power for aggressive action; and while such a policy may be sound and justifiable under existing conditions it clearly offers no promise of social peace.

Payment by results or piecework may be considered as the final effort of an outworn system to justify itself. Superficially, it seems fair and reasonable in almost any of its many forms; actually, it operates to increase the individual time-energy units expended, while decreasing, through diluted currency the exchange value of each time-energy unit, and crediting to the banker and the financier nearly the whole value of increased efficiency. If this contention is questioned, a reference to the much greater purchasing power of labour in the Middle Ages admitted in such books as “The Six Hour Day”\footnotemark[1] must surely confirm it.

In actual practice piecework neither does nor can take into consideration that, just as there is no limit to progress either of method or dexterity, so is there no fundamental relation between money and value as at present understood.

Consequently, all piecework systems produce in varying degree one of three conditions, either

1. Large classes of workers earn continuously increasing sums of money which bear no ratio to equally meritorious efforts on other bases of payment. If any effort is made to unify the basis on a large scale the purchasing power of money becomes completely unstable.

or

2. A piece rate is “nursed” to avoid any urgent incentive to change of method as an excuse for cutting the rate and earnings, with the result that output is restricted to a locally agreed basis, having no relation to either real or effective demand.

or

3. The price will be cut periodically by dubious management, a constant state of friction engendered, and the whole affair surrounded with an atmosphere of suspicion.

These results are logical, and to blame any special interest for any of them is beside the point. The use-value of the product, short time, unemployment, to say nothing of the elemental facts of industrial psychology and economics, are not considered at all in such systems; with the result that the victims make, so far as Trade Unions on the one hand and Employers’ Federations on the other, can assist them, their own arrangements for protection against the more dire consequences of crude forms of scientific management, or lukewarm service.

We have now arrived at this position; we desire to produce a definite programme of necessaries with a minimum expenditure of time-energy units. We agree that the substitution of human effort by natural forces through the agency of machinery is the clear path to this end; and we require to correlate to this a system which will arrange for the equitable distribution of the whole product while, at the same time, providing the most powerful incentive to efficiency possible.

The general answer to this problem may be stated in the four following propositions, which represent an effort to arrive at the Just Price:-—

\begin{enumerate}
	\item Natural resources are common property, and the means for their exploitation should also be common property.


	\item The payment to be made to the worker, no matter what the unit adopted, is the sum necessary to enable him to buy a definite share of ultimate products irrespective of the time taken to produce them.


	\item The payment to be made to the improver of process, including direction, is to be based on the rate of decrease of human time-energy units resulting from the improvement, and is to take the form of an extension of facilities for further improvement in the same or other processes.


	\item Labour is not exchangeable; product is.



\end{enumerate}
No attempt will be made to prove these propositions since their validity rests on equity.

It should be noted particularly that none of these points has any relation to systems of administration, although a recognition of them would radically affect the distribution of personnel in any system of administration.

While the distribution of the product of industry is fundamentally involved, and the inducements to vary the articles produced are clearly modified to a degree which would profoundly alter the industrial situation, no extension of bureaucracy in the accepted sense is implied or induced.

It may be argued that these principles are not susceptible of immediate embodiment; but it is, nevertheless, well to bear in mind the imminence of an economic breakdown (as a direct result of the inflation of currency by the capitalisation of negative values) already discussed, and the probability that a new economic system, having as its basis the principles of the law of the conservation of energy, will replace it.

It may be said in regard to proposition 1 that it involves a confiscation of plant, which is clearly an injustice to the present owners. But is it?

A reference to the accounting process already described will make it clear \emph{that the community has already bought and paid for many times over the whole of the plant used for manufacturing processes}, the purchase price being included in the selling price of the articles produced, and representing, in the ultimate, effort of some sort, but immediately, a rise in the cost of living. If the community can use the plant it is clearly entitled to it, quite apart from the fact that under proper conditions there is no reason why every reasonable requirement of its present owners should not be met under the changed conditions.

Before allowing the methods of compromise (which may or may not be desirable in the practicable evolution of a better conception of the community based on these propositions) to obscure the objective, a purely idealistic interpretation of them may be worth consideration, as a basis from which to deduce a practical policy.

Let us imagine the theories of rent and wages to be swept away and discredited, the existing industrial plant to be the property of the community and to be operating with technical efficiency. We are in possession of a census of the material requirements of the community, and are producing to a programme either based on those requirements or on the indirect achievement of them by the processes of barter with similar communities.

Since no extension or alteration of this programme is possible without affecting the whole community, the administration of real capital, \emph{i.e.}, \emph{the power to draw on the collective potential capacity to do work}, is clearly subject to the control of its real owners through the agency of credit.

Let us imagine this collective credit organisation, which might preferably not be the State, to be provided with the necessary organisation to fit it to pass upon, and if desirable to sanction, any private enterprise deemed to be in the interest of the community represented, the necessary capitalisation being secured by the general credit. It is clear that such an arrangement involves an appraisal of values both in respect to persons and materials, but it does not necessarily involve any control of policy whatever in respect of the internal administration of any undertaking once originated.

Under these conditions the community can be regarded as a single undertaking (decentralised as to administration to any extent necessary) and every individual comprised within it is in the position of an equal Bondholder entitled to an equal share of product. The distribution of the product is simply a problem of the arbitrary adjustment of prices to fit the dimensions of a periodical order to pay, issued to each bondholder, and it will be found that such prices will normally be less than cost, as measured by existing methods.

Let this annual order to pay be inalienable but carrying the assumption that a definite percentage of the individual’s stock of time-energy units is freely placed at the disposal of the community. Let these time-energy units be graded so that the lowest grade represents the poorest capacity multiplied by the time-factor, and let all adults on entering productive industry be so graded, and let the least attractive work be done by the agency of these time-energy units. Let an improvement of grade be based on the proposal by the individual of methods, processes, or organisation, resulting in a diminution of the total time-energy units required for the programme of production, and the success of the proposals. (It will be noticed that the strongest incentive to right judgment as regards facilities for trial exists here.) Let the possession of a definite “grade” of time-energy units be the absolute qualification for each class of employment; that is to say, proved ability to render special service will be the qualification for facilities to render service, but will not affect the division of product.

Now, it will be noticed that we have under these conditions absolute equity both personal and social. All improvement in process is to the general benefit, while, at the same time, the psychological reward of specific ability is exactly that which common experience shows to be the most perfectly satisfactory. No questions of material remuneration enter into the problem of administration at all; and increased complexity of manufactured product is either bought by increased efficiency or longer working hours; while simplicity of life provides greater opportunities for the use of the product and other activities. A system not dissimilar from the existing Shop Steward system, but with its members acting in the rôle of Citizens and not as Artisans, might control \emph{policy} absolutely, \emph{i.e.}, increase or decrease programmes of production and efficiency, etc., without interfering or having any possible incentive to interfere in direction or function. Economic incentive to competition other than in efficiency would disappear completely, and with it the primary cause of war.

\footnotetext[1]{“The Six Hour Day and other Industrial Problems.”–Lord Lever Hulme.

}\chapter{Chapter Nine}
\label{chapter-9}
While a much higher development not only of civic sense but of material progress is necessary to any realisation of a scheme of society based on anything approximating to the foregoing sketch, it is quite probable that eventually such an arrangement might be the only solution having inherent stability.

But a transition period is highly desirable, and as the present structure is susceptible of change by metabolism, it may be well to consider one of the numerous expedients available to that end.

Since an immediate levelling up of real purchasing power is absolutely essential if industry is to be kept going at all, the first point on which to be perfectly clear is that increasing wages on the grand scale is simply childish. Given a minimum percentage of profit and a fixed process, under the existing economic system the real wage, in the sense of a proportion of product, is steadily decreasing; and nothing will alter that fact except change of process (temporarily) and change of economic system (permanently). Even taxation of profits is quite incapable of providing any real remedy, because, as we have seen, the sum of the wages, salaries and dividends distributed in respect of the world’s production, even if evenly distributed, would not buy it, since the price includes non-existent values. There is no doubt whatever that the first step towards dealing with the problem is the recognition of the fact that what is commonly called credit by the banker is administered by him primarily for the purpose of private profit, whereas it is most definitely communal property. In its essence it is the estimated value of the only real capital—it is the estimate of the \emph{potential} capacity under a given set of conditions, including plant, etc., of a Society to do work. The banking system has been allowed to become the administrator of this credit and its financial derivatives with the result that the creative energy of mankind has been subjected to fetters which have no relation whatever to the real demands of existence, and the allocation of tasks has been placed in unsuitable hands.

Now it cannot be too clearly emphasised \emph{that real credit is a measure of the reserve of energy belonging to a community and in consequence drafts on this reserve should he accounted for by a financial system which reflects that fact}.

If this be borne in mind, together with the conception of “Production” as a conversion, absorbing energy, it will be seen that the individual should receive something representing the diminution of the communal credit-capital in respect of each unit of converted material.

It remains to consider how these abstract propositions can be given concrete form.

So far as this country is concerned, the instrument which comes most easily to the hand to deal with the matter is the National Debt, which for practical purposes may be considered to be the War Debt in all its forms, although it should be clearly understood that all appropriations of credit can be considered as equally concerned.

Some consideration of the real nature of the debt is necessary in order to understand the basis of this proposal.

The £8,000,000,000 in round numbers which have been subscribed for war purposes represents as to its major portion (apart from about £1,500,000,000 re-lent) services which have been rendered and paid for, and in particular, the sums paid for munitions of all kinds, payment of troops and sums distributed in pensions and other doles. Now, the services have been rendered and the munitions expended, consequently, the loan represents a lien with interest on the future activities of the community, in favour of the holders of the loan, that is to say, the community guarantees the holders to work for them without payment, for an indefinite period in return for services rendered by the subscribers to the Loan. What are those services?

Disregarding holdings under £1,000 and re-investment of pre-war assets, the great bulk of the loan represents purchases by large industrial and financial undertakings \emph{who obtained the money to buy by means of the creation and appropriation of credits at the expense of the community, through the agency of industrial accounting and bank finance}.

It is not necessary to elaborate this contention at any great length because it is quite obviously true. Eventually, to have any meaning, the loan must be paid off in purchasing power over goods not yet produced, and is, therefore, simply a portion of the estimated capacity of the nation to do work which has been hypothesized.

Whatever may be said of subscriptions out of wages and salaries, therefore, there is not the slightest question that in so far as the loan represents the capitalisation of the processes already described, its owners have no right in equity to it—it simply represents communal credit transferred to private account.

To put the matter another way: For every shell made and afterwards fired and destroyed, for every aeroplane built and crashed, for all the stores lost, stolen or spoilt, the Capitalist has an entry in his books which he calls wealth, and on which he proposes to draw interest at 5\%, whereas that entry represents loss not gain, debt not credit, to the community, and, consequently, is only realisable by regarding the interest of the Capitalist as directly opposite to that of the community. \emph{Now, it must be perfectly obvious to anyone who seriously considers the matter that the State should lend, not borrow, and that in this respect, as in others, the Capitalist usurps the function of the State}.

But, however the matter be considered, the National Debt as it stands is simply a statement that an indefinite amount of goods and services (indefinite because of the variable purchasing power of money) are to be rendered in the future to the holders of the loan, \emph{i.e.}, it is clearly a distributing agent.

Now, instead of the levy on capital, which is widely discussed, let it be recognised that credit is a communal, not a bankers’ possession; let the loan be redistributed by the same methods suggested in respect of a capital levy so that no holding of over £1,000 is permitted; to the end that, say, 8,000,000 heads of families are credited with £50 per annum of additional purchasing power.

And further, let all production be costed on a uniform system open to inspection, the factory cost being easily ascertained by making all payments through a credit agency; the manner of procedure to this end is described hereafter. Let all payments for materials and plant be made through the Credit Agency and let plant increases be a running addition to the existing National Debt, and let the yearly increase in the debt be equally distributed after proper depreciation. Let the selling price of the product be adjusted in reference to the effective demand by means of a depreciation rate fixed on the principle described subsequently, and let all manufacturing and agriculture be done, with broad limits, to a programme. Payment for industrial service rendered should be made somewhat on the following lines:-—

Let it be assumed that a given production centre has a curve of efficiency varying with output, which is a correct statement for a given process worked at normal intensity. The centre would be rated as responsible for a programme over a given time such that this efficiency would be a maximum when considered with reference to, say, a standard six-hour day. On this rating it is clear that the amount of money available for distribution in respect of labour and staff charges can be estimated by methods familiar to every manufacturer.

Now let this sum be allocated in any suitable proportion between the various grades of effort involved in the undertaking, and let a considerable bonus together with a recognised claim to promotion be assured to any individual who by the suggestion of improved methods or otherwise, can for the specified programme, reduce the hours worked by the factory or department in which he is engaged.

Now, consider the effect of these measures: Firstly, there is an immediate fall in prices which is cumulative, and, consequently, a rise in the purchasing power of money. Secondly, there is a widening of effective demand of all kinds by the wider basis of financial distribution. There is a sufficient incentive to produce, but there is communal control of undesirable production through the agency of credit; and there is incentive to efficiency. There is the mechanism by which the most suitable technical ability would be employed where it would be most useful, while the separation of a sufficient portion of the machinery of economic distribution from the processes of production would restore individual initiative, and, under proper conditions, minimise the effects of bureaucracy.

This rapid survey of the possibilities of a modified economic system will, therefore, probably justify a somewhat more detailed examination of certain features of the proposed structure, and clearly the control and use of credit is of primary importance. It should be particularly noted at this point, however, that every suggestion made in this connection has in view the maximum expansion of personal control of initiative and the minimising and final elimination of economic domination, either personal or through the agency of the State.

\chapter{Chapter Ten}
\label{chapter-10}
IN considering the inadequacy of a mere extension of manufacturing production unaccompanied by a modification of the distributing system, it was seen that in any manufacturing process there enters into the cost, and re-appears in the price, a charge for certain items which are really rendered useless, but which form a step towards the final product. These items may be conveniently grouped under the heading of semi-manufactures when considered in relation to a more complex product, although in many cases they may in themselves, for other purposes, represent a final product. For instance, electric power, if used for lighting, is a final product, and ministers directly to a human need, but the same energy, if used to drive a cotton mill, is in the sense in which the term is here used, a semi-manufacture.

Now, it should be obvious that a semi-manufacture in this sense is of no use to a consumer—if it is used as an ultimate product it ceases to come under the heading of a semi-manufacture.

Therefore, a semi-manufacture must be an asset to be accounted into an estimate of the potential capacity to produce ultimate products (which is the whole object of manufacture from a human point of view), and with certain reservations represents an increase of credit-capital but not of wealth. This conception is of the most fundamental importance.

If we concede its validity, a transfer of value in respect of semi-manufactures as between one undertaking and another is measured by a transfer of real credit, and like a financial credit transfer is most suitably dealt with through the agency of a Clearing-house.

Let us imagine such a Clearing-house to exist and endeavour to analyse its operations in respect to Messrs. Jones and Company who tan leather, Messrs. Brown and Company who make boots, and Messrs. Robinson who sell them, and let us imagine that all these undertakings are run on the basis of a commission or profit on all labour and salary costs, an arrangement which is, however, quite immaterial to the main issue.

Messrs. Jones receives raw hides of the datum value of £100 which require semi-manufactures value £500 to turn out as leather, together with the expenditure of £500 in wages and salaries. Messrs. Jones order the hides and the semi-manufactures by the usual methods from any source which seems to them desirable, and on receipt of the invoices, turn these into the Clearing-house, which issues a cheque in favour of Messrs. Jones for the total amount;£600; by means of which Messrs. Jones deal with their accounts for supplies.

The Clearing-house writes \emph{up} its capital account by this sum, and by all sums issued by it. The out-of-pocket cost to Messrs. Jones of their finished product is, therefore, £500. Let us allow them 10\%. profit on this, and the cost, plus profit, at the factory under these conditions is £550, and a sum of £600 is owing to the Clearing-house.

Messrs. Brown who require these hides for boot-making, order them from Messrs. Jones, and other supplies from elsewhere amounting to £500, and similarly transmit Messrs. Jones’ invoices (which include the sums paid by the Clearing-house) with the rest to the Clearing-house, which issues a cheque for £l,650 to Messrs. Brown, who pay Messrs. Jones; who, in turn, retain £550 and pay back £600 to the Clearing-house. Messrs. Jones are now disposed of. They have made their own arrangements in respect of quantity of labour, etc., and have made a profit of 10\%, on the cost of this labour.

Messrs. Brown now make the leather into boots, expending a further £500 in salaries and wages, and making 10\%, profit on this. They receive an order from Messrs. Robinson for these boots: and Messrs. Robinson’s own out-of-pocket cost, with their commission, is £300 paid by a cheque from the Clearing-house for £2,200 + £300, £2,200 of which goes to Messrs. Brown, who pay off their debt of £1,650 and retain the remainder.

Now let us notice that the purchasing power released externally in these transactions is that represented by wages, salaries and a commission on them, and that no goods have been yet released to consumers against this purchasing power. These sums thus distributed will be largely expended by the recipients in various forms of consumption, and it is only their joint surplus which will assist in providing an effective demand for Messrs. Robinson’s stock. The price of this stock then requires adjustment.

Let us now introduce into the transactions a document we may call a retail clearing invoice, which might form in its description of the goods a duplicate of the bill paid by the purchaser of an article for the purpose of ultimate consumption; and let it be understood that a properly executed retail clearing invoice is accepted by the Clearing-house as evidence of the transfer of goods to an actual consumer. It will be seen that by the process previously explained we have distributed the means of purchase and are left in a position to fix the price without reference to the individual interests of Messrs. Brown, Jones or Robinson, as so far the cost is charged to capital account. The question is what should the price be? The answer to this is a \emph{statement of the average depreciation of the capital assets of the community, stated in terms of money released over an equal period of time, and the correct price is the money value of this depreciation in terms of the cost of the article}. In other words, the Just Price of an article, which is the price at which it can be effectively distributed in the community producing, bears the same ratio to the cost of production that the total consumption and depreciation of the community bears to the total production.

Let us now apply this to our example of such a staple as the supply of boots.

Let us assume that in a given credit area the sum involved in the delivery of boots to the user per month amounts to £2,500, that is to say, the \emph{cost} figures of the retail invoices turned into the Clearing-house per month total that sum. This means that services have been rendered and remunerated by the payment over an indefinite period of the token value of £2,500, and the product of these services distributed in one month. But the token value has a general purchasing power, consequently, it should be set against a general value. The general value is equal to the general rate of depreciation, or if it be preferred, consumption of the whole of the goods which can be bought with the token value. Let us assume this to be 40\%, that is to say, let us imagine that of the total work of the community for one month 60\%, remains for use during a subsequent period. Then the selling price of a pair of boots would be equal to 40\%, of £2,500 divided by the total number of pairs of boots distributed (not pairs produced); or would be ⅖ \emph{of commercial cost}. Messrs. Robinson, therefore, in respect of £2,500 of retail invoices turned in by them (which would include their own labour and commission) would be credited with 60\% of that sum against the cheque originally sent them (out of which they paid Messrs. Brown) recovering the remaining 40\%, from the actual purchasers of the boots, and reimbursing the Clearing-house; who after balancing Messrs. Robinson’s account would write \emph{down} their own credits by that amount. This would leave the credit-capital of the community—that is to say, the financial estimate of potential capacity to deliver goods—written up by 60\% of £2,500, which is an accounting reflection of the actual situation.

From this point of view, all semi-manufactures become simply a form of tool power, and are subject to the same treatment as manufacturing plant; they are a form of capital assets to be depreciated and written down from time to time. There is absolutely no difference in principle between the treatment in this manner of a tool which wears out in five years’ time and a unit of energy which is dissipated in a few minutes in driving the tool.

We arrive, then, at a conception of credit employment, by which all semi-manufacturers are treated as additions to communal capital account; subject to writing down as they are actually consumed as ultimate products. In order to be effective the writing down must take the form of a cancellation of credit-capital, a process which is done quite simply and automatically by the application to the capital account of retail clearing invoices in the manner roughly outlined, or by any other device which is based on the dynamic conception of industry.

Exactly the same treatment is applicable to the installation of fresh tools, buildings, etc., although for convenience, no doubt, separate accounts for such assets would be desirable, since the writing down would be done at somewhat longer intervals.

We have now clearly arrived at a point where there is a direct relation between effective demand and prices, as distinct from the relation between costs and prices. Let us now imagine a single adjustable tax applied to all production, of such magnitude as to bring prices from those fixed by the foregoing method to the suitable international exchange level. In existing circumstances, without affecting present prices, such a tax would pay the interest on the War Loan many times over. Let such a tax be applied to this purpose, the War Loan being distributed in the manner described and possibly increased by additions from Clearing-house transfers. It is clear that a rise in external prices would be met by an increased distribution, while a greater internal efficiency would have a similar result. Such an arrangement would make it possible to effect, in fact, would certainly induce, a transition from a purely competitive world system to one exhibiting in concrete form the demand for cooperation without regimentation, which, beyond all question, underlies the so-called proletarian revolt.

It may, perhaps, at this juncture, be desirable to emphasise the obvious, to the extent of pointing out that no financial system by itself affects concrete facts; that the object of measures of the character indicated is the provision of the right incentive to effort and the removal of any possible incentive to waste; and only to the extent that these are achieved is the economic emancipation of the individual brought nearer to reality. Had the principles underlying these suggestions been generally understood and accepted during the war, we should have experienced a steady decrease of purchasing power by every individual, which would have enabled us to resume the general improvement in social conditions at its close, without that misunderstanding of facts which now threatens catastrophe. The depreciation rate would, in a manner quite similar to that with which we are familiar in the case of the Bank rate, have been raised at suitable intervals to represent the excess of destruction over production; the necessity of increased effort would have been brought home to every individual by decreased distribution in respect of National Capital assets, and the general atmosphere of distrust and recrimination, from which we suffer as a result of confusion of thought, would probably not have arisen.

\chapter{Chapter Eleven}
\label{chapter-11}
THE awful tragedy of waste and misery through which the world has passed during the years 1914-1919 has brought about a widespread determination that the best efforts of which mankind is capable are not too much to devote to the construction of a fabric of society within which a repetition of the disaster would be, if not impossible, unlikely; and the major focus of this determination has found a vehicle in the project commonly known as the League of Nations.

The immense appeal which the phrase has made to the popular and honest mind has made it dangerous to fail in rendering lip service to it; but it is fairly certain that under cover of the same form of words one of the most gigantic and momentous struggles in history is waged for the embodiment of either of the opposing policies already discussed.

The success of an attempt to impose an economic and political system on the world by means of armed force would mean the culmination of the policy of centralised control, and the certainty that all the evils, which increasing centralisation of administrative power has shown to be inherent in a power basis of society, would reach in that event their final triumphant climax.

But there is no final and inevitable relation between the project of international unity and the policy of centralised control. Just as in the microcosm of the industrial organisation there is no difficulty in conceiving a condition of individual control of policy in the common interest, so in the larger world of international interest the character and effect of a League of Free Peoples is entirely dependent on the structure by which those interests which individuals have in common can be made effective in action.

Now, unless the earlier portions of this book have been written in vain, it has been shown that the basis of power in the world to-day is economic, and that the economic system with which we are familiar is expressly designed to concentrate power. It follows inevitably from a consideration of this proposition that a League of Nations involving centralised military force is entirely interdependent upon the final survival of the Capitalistic system in the form in which we know it, and conversely that the fall of this system would involve a totally different international organisation. A superficial survey of the position would no doubt suggest that the triumph of central control was certain; that the power of the machine was never so great; and that, whether by the aid of the machine-gun or mere economic elimination, the scattered opponents to the united and coherent focus of financial and military power would within a measurable period be reduced to complete impotence and would finally disappear.

But a closer examination of the details tends to modify that view, and to confirm the statement already made that a pyramidal administrative organisation, though the strongest against external pressure, is of all forms the most vulnerable to disruption from within.

We have already seen that a feature of the industrial economic organisation at present is the illusion of international competition, arising out of the failure of internal effective demand as an instrument by means of which production is distributed. This failure involves the necessity of an increasing export of manufactured goods to undeveloped countries, and this forced export, which is common to all highly developed capitalistic States, has to be paid for almost entirely by the raw material of further exports. Now, it is fairly clear that under a system of centralised control of finance such as that we are now considering, this forced competitive export becomes impossible; while at the same time the share of product consumed inside the League becomes increasingly dependent on a frenzied acceleration of the process.

The increasing use of mechanical appliances, with its capitalisation of overhead charges into prices, renders the distribution of purchasing power, through the medium of wages in particular, more and more ineffective; and as a result individual discontent becomes daily a more formidable menace to the system. It must be evident therefore that an economic system involving forced extrusion of product from the community producing, as an integral component of the machinery for the distribution of purchasing power, is entirely incompatible with any effective League of Nations, because the logical and inevitable end of economic competition is war. Conversely, an effective League of Free Peoples postulates the abolition of the competitive basis of society, and by the installation of the cooperative commonwealth in its place makes of war not only a crime, but a blunder.

Under such a modification of world policy, inter-change of commodities would take place with immeasurably greater freedom than at present, but on principles exactly opposite to those which now govern Trade. The manufacturing community now struggles for the privilege of converting raw material into manufactured goods for export to less developed countries. Non-competitive industry would largely leave the trading initiative to the supplier of raw material. Since any material received in payment of exported goods would find a distributed effective demand waiting for it, imports would tend to consist of a much larger proportion of ultimate products for immediate consumption than is now the case; thus forcing on the more primitive countries the necessity of exerting native initiative in the provision of distinctive production.

Again, International legislation in regard to labour conditions under a competitive system must always fail at the point at which it ceases to be merely negative, because it has ultimately to consider employment as an agency of distribution, and rightly considered distribution should be a function of work accomplished, not of work in progress, \emph{i.e.}, employment. As a consequence, this most important field of constructive effort resolves itself into a battleground of opposing interests, both of which are merely concerned with an effort to get something for nothing. The inevitable compromise can be in no sense a settlement of such questions, any more than the succession of strikes for higher pay and shorter hours, which are based on exactly the same conception, can possibly result in themselves in a stable industrial equilibrium.

Examples of the same class of difficulty might be multiplied indefinitely, but enough has probably been said to indicate the disruptive nature of the forces at work. To state whether or not the general confusion and misdirection of opinion will make a period of power control inevitable, in order to unite public opinion against it, would be to venture into a form of prophecy for which there is no present justification; but it is safe to say, that whether after the lapse of a few months, or of a very few years, the conception of a world governed by the concentrated power of compulsion of any description whatever, will be finally discredited and the instruments of its policy reduced to impotence.

\chapter{Chapter Twelve}
\label{chapter-12}
As a result of the survey of the wide field of unrest and the attempt to analyse, and as far as possible to simplify, the common elements which are its prime movers, it appears probable that the concentration of economic power through the agency of the capitalistic system of price fixing, and the control of finance and credit, is of all causes by far the most immediately important and therefore that the distribution of economic power back to the individual is a fundamental postulate of any radical improvement. While this, it would seem, is indisputable, it must not be assumed that by the attainment of individual economic independence, the social problems which are so menacing, would immediately disappear. The reproach is frequently levelled at those who insist on the economic basis of society that in them materialism is rampant, and in consequence the bearing of sentiment on these matters is overlooked, and the immense and decisive influence on events which is exerted by such factors is very apt to be ignored. There is a germ of truth in this; but if such critics will consider the origin of popular sentiment, the influence of economic power will be seen to predominate in this matter also, whether considered merely as the tool of a policy, or as an isolated phenomenon.

It is claimed, and more particularly by those who utilise it, that “public opinion” is the decisive power in public affairs. Assuming that in some sense this may be true, it becomes of interest to consider the nature of this public opinion and the basis from which it proceeds, and it will be agreed that the chief factors are education and propaganda.

Now, the bearing of economic power on education hardly requires emphasis. In England, the Public School tradition, with all its admirable features, is nevertheless an open and unashamed claim to special privilege based on purchasing power and on nothing else; and with a sufficient number of exceptions, its product is pre-eminently efficient in its own interest, as distinct from that of the community. It is one of the most hopeful and cheering features of the present day that this defect is increasingly deplored by all the best elements comprised within the system; and the danger of reaction in the future is to that extent reduced.

But by far the most important instrument used in the moulding of public opinion is that of organised propaganda either through the Public Press, the orator, the picture, moving or otherwise, or the making of speeches; and in all these the mobilising capacity of economic power is without doubt immensely if not preponderatingly important.

When it is considered that the expression of opinion inimical to “vested interests” has in the majority of cases to be done at the cost of financial loss and in the face of tremendous difficulty, while a platform can always be found or provided for advocates of an extension of economic privilege, the fundamental necessity of dealing \emph{first} with the economic basis of society must surely be, and in fact now is, recognised, and this having been established in conformity with a considered policy the powers of education and propaganda will be free from the improper influences which operate to distort their immense capacity for good.

The policy suggested in the foregoing pages is essentially and consciously aimed at pointing the way, in so far as it is possible at this time, to a society based on the unfettered freedom of the individual to cooperate in a state of affairs in which community of interest and individual interest are merely different aspects of the same thing. It is believed that the material basis of such a society involves \emph{the administration of credit by a decentralised local authority; the placing of the control of process entirely in the hands of the organised producer} (and this in the broadest sense of the evolution of goods and services) and the \emph{fixing of prices on the broad principles of use value, by the community as a whole operating by the most flexible representation possible}.

On such a basis, the control of the sources of information in the interests of any small section of the community becomes an anomaly without a specific meaning; and the prostitution of the Press and of similar organs of publicity would no doubt within a measurable time disappear because it would lack objective. But there would still remain the task of eradicating the hypnotic influence of a persistent presentation of distorted information, at any rate so far as this generation of humanity is concerned, and it seems clear that a radical and democratic basis of Publicity control is an integral factor in the production of the better society on which the Plain People have quite certainly determined.

Thus out of threatened chaos might the

Dawn break; a Dawn which at the best

must show the ravages of storm,

but which holds clear for all

to see the promise of

a better Day.



\end{document}
